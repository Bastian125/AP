\section{Auswertung}
\label{sec:Auswertung}

\subsection{Vorbereitungsaufgabe}
\label{sec:Vorbereitungsaufgabe}
Die benötigten Literaturwerte der Absorbermaterialien sind in \autoref{tab:literatur} zu finden.
\begin{table}
  \centering
  \caption{Literaturwerte.}
  \label{tab:literatur}
  \begin{tabular}{c | c c c c}
    \toprule
    & $Z$ & $E_{\symup{K}}^{\symup{Lit}}/\unit{\kilo\electronvolt}$ & $\theta_{\symup{K}}^{\symup{Lit}}$ & $\sigma_{\symup{K}}^{\symup{Lit}}$ \\
    \midrule
    $\symup{Zn}$  & 30 &  9,65 & 18,60 & 3,56 \\
    $\symup{Ge}$  & 32 & 11,10 & 16,12 & 3,68 \\
    $\symup{Br}$  & 35 & 13,47 & 13,23 & 3,85 \\
    $\symup{Rb}$  & 37 & 15,20 & 11,70 & 3,94 \\
    $\symup{Sr}$  & 38 & 16,10 & 11,04 & 4,00 \\
    $\symup{Zr}$  & 40 & 17,99 &  9,86 & 4,10 \\
    $\symup{Ga}$  & 31 & 10,37 & 17,29 & 3,61 \\
    \bottomrule
  \end{tabular}
\end{table}
Die Literaturwerte für die Kupferröntgenröhre sind
\begin{align*}
  E_{\symup{K}, \alpha} &= 8\,\unit{\kilo\eV}\;, \\
  E_{\symup{K}, \beta}  &= 8,95\,\unit{\kilo\eV}\;, \\
  \theta_{\symup{K}, \alpha} &= 22,66^{\circ}\; \symup{und} \\
  \theta_{\symup{K}, \beta} &= 20,14^{\circ}.
\end{align*}

\subsection{Bragg-Bedingung}
\label{sec:Bragg}

Die Messwerte zur Überprüfung der Bragg-Bedingung sind in \autoref{tab:Bragg} zu finden und in \autoref{fig:bragg}
dargestellt. Es wurde ein Maxium bei $27,7^{\circ}$ festgestellt. Nachdem Reflexionsgesetz liegt der
theoretische Winkel bei $28^{\circ}$. Daraus ergibt sich eine Abweichung von $1,5\%$.

\begin{table}
  \centering
  \begin{tabular}{c c | c c}
    \toprule
    $\theta^{\circ}$ & $N/Imp/s$ & $\theta^{\circ}$ & $N/Imp/s$ \\
    \midrule
    26,0 &  39,0 & 28,2 & 272,0 \\
    26,1 &  43,0 & 28,3 & 263,0 \\
    26,2 &  43,0 & 28,4 & 255,0 \\
    26,3 &  52,0 & 28,5 & 247,0 \\
    26,4 &  76,0 & 28,6 & 234,0 \\
    26,5 &  85,0 & 28,7 & 236,0 \\
    26,6 & 113,0 & 28,8 & 222,0 \\
    26,7 & 117,0 & 28,9 & 206,0 \\
    26,8 & 146,0 & 29,0 & 181,0 \\
    26,9 & 164,0 & 29,1 & 185,0 \\
    27,0 & 183,0 & 29,2 & 164,0 \\
    27,1 & 182,0 & 29,3 & 155,0 \\
    27,2 & 216,0 & 29,4 & 154,0 \\
    27,3 & 219,0 & 29,5 & 129,0 \\
    27,4 & 238,0 & 29,6 & 110,0 \\
    27,5 & 256,0 & 29,7 & 100,0 \\
    27,6 & 281,0 & 29,8 &  90,0 \\
    27,7 & 277,0 & 29,9 &  84,0 \\
    27,8 & 274,0 & 30,0 &  75,0 \\
    27,9 & 269,0 & & \\
    28,0 & 278,0 & & \\
    28,1 & 274,0 & & \\
    \bottomrule
  \end{tabular}
  \caption{Messwerte zur Überrüfung der Bragg-Bedingung.}
  \label{tab:Bragg}
\end{table}

\begin{figure}
  \centering
  \includegraphics{bragg.pdf}
  \caption{Werte zur Bestimmung der Bragg-Bedingung.}
  \label{fig:bragg}
\end{figure}

\subsection{Emissionsspektrum der Cu-Röntgenröhre}
\label{sec:cu}
Die Messwerte des Emissionsspektrums der Kupferröntgenröhre sind in \autoref{tab:Cu} dargestellt. Weiterhin sind Die
Messwerte des Detailspektrums in \autoref{tab:Detailspektrum} zu finden.

\begin{table}
  \centering
  \begin{tabular}{c c | c c | c c}
    \toprule
    $\theta^{\circ}$ & $N/Imp/s$ & $\theta^{\circ}$ & $N/Imp/s$ & $\theta^{\circ}$ & $N/Imp/s$ \\
    \midrule
     4,0 &  40,0 & 11,2 & 409,0 & 18,6 &  176,0 \\
     4,2 &  42,0 & 11,4 & 411,0 & 18,8 &  158,0 \\
     4,4 &  38,0 & 11,6 & 403,0 & 19,0 &  161,0 \\
     4,6 &  34,0 & 11,8 & 393,0 & 19,2 &  154,0 \\
     4,8 &  37,0 & 12,0 & 381,0 & 19,4 &  152,0 \\
     5,0 &  60,0 & 12,2 & 385,0 & 19,6 &  150,0 \\
     5,2 &  70,0 & 12,4 & 389,0 & 19,8 &  200,0 \\
     5,4 & 113,0 & 12,6 & 389,0 & 20,0 & 1374,0 \\
     5,6 & 130,0 & 12,8 & 374,0 & 20,2 & 1437,0 \\
     5,8 & 144,0 & 13,0 & 370,0 & 20,4 & 1021,0 \\
     6,0 & 167,0 & 13,2 & 351,0 & 20,6 &  230,0 \\
     6,2 & 189,0 & 13,4 & 322,0 & 20,8 &  205,0 \\
     6,4 & 201,0 & 13,6 & 294,0 & 21,0 &  185,0 \\
     6,6 & 221,0 & 13,8 & 277,0 & 21,2 &  177,0 \\
     6,8 & 238,0 & 14,0 & 273,0 & 21,4 &  169,0 \\
     7,0 & 273,0 & 14,2 & 270,0 & 21,6 &  179,0 \\
     7,2 & 279,0 & 14,4 & 263,0 & 21,8 &  194,0 \\
     7,4 & 299,0 & 14,6 & 261,0 & 22,0 &  269,0 \\
     7,6 & 296,0 & 14,8 & 244,0 & 22,2 & 2417,0 \\
     7,8 & 304,0 & 15,0 & 250,0 & 22,4 & 4662,0 \\
     8,0 & 329,0 & 15,2 & 247,0 & 22,6 & 4345,0 \\
     8,2 & 335,0 & 15,4 & 244,0 & 22,8 & 1292,0 \\
     8,4 & 333,0 & 15,6 & 237,0 & 23,0 &  182,0 \\
     8,6 & 351,0 & 15,8 & 244,0 & 23,2 &  142,0 \\
     8,8 & 366,0 & 16,0 & 224,0 & 23,4 &  124,0 \\
     9,0 & 377,0 & 16,2 & 221,0 & 23,6 &  114,0 \\
     9,2 & 375,0 & 16,4 & 224,0 & 23,8 &  109,0 \\
     9,4 & 385,0 & 16,6 & 218,0 & 24,0 &  104,0 \\
     9,6 & 376,0 & 16,8 & 203,0 & 24,2 &  103,0 \\
     9,8 & 395,0 & 17,0 & 197,0 & 24,4 &   91,0 \\
    10,0 & 414,0 & 17,2 & 200,0 & 24,6 &   87,0 \\
    10,2 & 416,0 & 17,4 & 196,0 & 24,8 &   91,0 \\
    10,4 & 408,0 & 17,6 & 188,0 & 25,0 &   84,0 \\
    10,6 & 417,0 & 17,8 & 186,0 & 25,2 &   81,0 \\
    10,8 & 403,0 & 18,0 & 170,0 & 25,4 &   77,0 \\
    11,0 & 424,0 & 18,2 & 174,0 & 25,6 &   78,0 \\
    11,2 & 409,0 & 18,4 & 165,0 & 25,8 &   76,0 \\
         &       &      &       & 26,0 &   79,0 \\
    \bottomrule
  \end{tabular}
  \caption{Messwerte der Kupferröntgenröhre.}
  \label{tab:Cu}
\end{table}

\begin{table}
  \centering
  \begin{tabular}{c c | c c}
    \toprule
    $\theta^{\circ}$ & $N/Imp/s$ & $\theta^{\circ}$ & $N/Imp/s$ \\
    \midrule
    19,0 &  158,0 & 21.4 &  170,0 \\
    19,1 &  154,0 & 21.5 &  175,0 \\
    19,2 &  164,0 & 21.6 &  182,0 \\
    19,3 &  149,0 & 21.7 &  198,0 \\
    19,4 &  151,0 & 21.8 &  187,0 \\
    19,5 &  145,0 & 21.9 &  218,0 \\
    19,6 &  157,0 & 22.0 &  263,0 \\
    19,7 &  173,0 & 22.1 &  590,0 \\
    19,8 &  200,0 & 22.2 & 2305,0 \\
    19,9 &  688,0 & 22.3 & 4041,0 \\
    20,0 & 1392,0 & 22.4 & 4691,0 \\
    20,1 & 1473,0 & 22.5 & 4443,0 \\
    20,2 & 1439,0 & 22.6 & 4386,0 \\
    20,3 & 1413,0 & 22.7 & 3122,0 \\
    20,4 & 1040,0 & 22.8 & 1407,0 \\
    20,5 &  391,0 & 22.9 &  232,0 \\
    20,6 &  237,0 & 23.0 &  167,0 \\
    20,7 &  212,0 & 23.1 &  150,0 \\
    20,8 &  197,0 & 23.2 &  142,0 \\
    20,9 &  191,0 & 23.3 &  132,0 \\
    21,0 &  181,0 & 23.4 &  128,0 \\
    21,1 &  176,0 & & \\
    21,2 &  178,0 & & \\
    21,3 &  175,0 & & \\
    \bottomrule
  \end{tabular}
  \caption{Messwerte des Detailspektrums.}
  \label{tab:Detailspektrum}
\end{table}

\begin{figure}
  \centering
  \includegraphics{cu.pdf}
  \caption{Emissionsspektrum der Cu-Röntgenröhre.}
  \label{fig:cu}
\end{figure}
In \autoref{fig:cu} wird das Emissionspektrum dargestellt. Die $K_{\symup{\alpha}}$-, $K_{\symup{\beta}}$-Linie und
der Bremsberg werden aus den Messwerten durch Ablesen bestimmt. Die Halbwertsbreite (FWHM) bestimmt sich aus dem
Detailspektrum wieder durch Ablesen, so dass die Angabe eines statistischen Fehlers nicht nötig ist.
Die $K_{\symup{\alpha}}$-Linie befindet sich bei $2\theta=22,4^{\circ}$ und 
$N = 4662\,\frac{\symup{Imp}}{\unit{\second}}$. Die $K_{\symup{\beta}}$-Linie befindet sich bei 
$2\theta=20,2^{\circ}$ und $N = 1437\,\frac{\symup{Imp}}{\unit{\second}}$. Die Halbwertsbreiten bestimmen sich zu
\begin{align*}
  \theta_{\symup{FWHM, \alpha}} &= 0,55^{\circ} \; \symup{und} \\
  \theta_{\symup{FWHM, \beta}} &= 0,55^{\circ}. 
\end{align*}
Daraus und der Bragg-Bedingung aus \autoref{eqn:bragg} lassen sich nun die minimalen Wellenlängen bzw.
die maximale Energie $E = \frac{hc}{\lambda}$ bestimmen zu
\begin{align*}
  E_{K,\alpha}           &= 8,147\,\unit{\kilo\eV},\\
  E_{K, \beta}           &= 8,915\,\unit{\kilo\eV},\\
  \Delta E_{FWHM,\alpha} &= 0,233\,\unit{\kilo\eV} \; \symup{und}\\
  \Delta E_{FWHM,\beta}  &= 0,187\,\unit{\kilo\eV}.
\end{align*}
Die Abweichung von $E_{K,\alpha}$ zum Theoriewert beträgt $1,84\%$ und die für $E_{K, \beta}$ beträgt
$0,39\%$.
Das Auflösungsvermögen bestimmt sich durch
\begin{align*}
  A = \frac{E_{\symup{K}}}{\Delta E_{\symup{FWHM}}}.
\end{align*}
Für die $K_{\symup{\alpha}}$-Linie und $K_{\symup{\beta}}$-Linie bestimmt sich das Auflösungsvermögen zu
\begin{align*}
  A_{\symup{K}_{\alpha}}  &=  34,93 \; \symup{und}\\
  A_{\symup{K}_{\beta}}   &= 47,71.
\end{align*}
Die Absorptionskoeffizienten bestimmen sich durch \autoref{eqn:Ekabs}, \autoref{eqn:EKa} und
\autoref{eqn:EKb} zu
\begin{align*}
  \sigma_{\symup{1}} &= 3,3\;,\\
  \sigma_{\symup{2}} &= 13,35\; \symup{und}\\
  \sigma_{\symup{3}} &= 22,46.
\end{align*}

\subsection{Absorptionsspektren}
\label{sec:Absorptionsspektren}
Es werden die Absorptionsspektren von Brom, Gallium, Strontium, Zink und Zirkonium ausgewertet. Dazu werden die
Positionen der K-Kanten bestimmt. Die Abschirmkonstante kann durch
\begin{align*}
  \sigma_{\symup{K}} = z - \sqrt{\frac{E_{\symup{K}}}{R_{\infty}}-\frac{\alpha^2 \cdot Z^4}{4}}
\end{align*}
berechnet werden. Dabei ist $Z$ die Kernladungszahl und $\alpha$ die Feinstrukturkonstante und beträgt $7,297\cdot
10^{-3}$.
Die Messwerte der Absorber sind in \autoref{tab:bromGallium}, \autoref{tab:strontiumZink} und \autoref{tab:zirkonium} zu finden.
Die Werte sind in \autoref{fig:brom}, \autoref{fig:gallium}, \autoref{fig:strontium}, \autoref{fig:zink} und
\autoref{fig:zirkonium} dargestellt.
\begin{table}
  \centering
  \caption{Messwerte für Brom und Gallium}
  \label{tab:bromGallium}
  \begin{tabular}{c c | c c}
    \toprule
    $\theta_{\symup{Brom}}^{\circ}$ & $N_{\symup{Brom}}/Imp/s$ & $\theta_{\symup{Gallium}}^{\circ}$ & $N_{\symup{Gallium}}/Imp/s$ \\
    \midrule
    12,5 & 15 & 16,5  & 40 \\
    12,6 & 16 & 16,6  & 40 \\
    12,7 & 13 & 16,7  & 38 \\
    12,8 & 15 & 16,8  & 39 \\
    12,9 & 19 & 16,9  & 37 \\
    13,0 & 28 & 17,0  & 39 \\
    13,1 & 36 & 17,1  & 45 \\
    13,2 & 37 & 17,2  & 53 \\
    13,3 & 38 & 17,3  & 58 \\
    13,4 & 39 & 17,4  & 65 \\
    13,5 & 34 & 17,5  & 72 \\
    13,6 & 37 & 17,6  & 74 \\
    13,7 & 33 & 17,7  & 70 \\
    13,8 & 30 & 17,8  & 70 \\
    13,9 & 32 & 17,9  & 66 \\
    14,0 & 30 & 18,0  & 65 \\
    14,1 & 30 &       &    \\
    14,2 & 28 &       &    \\
    14,3 & 30 &       &    \\
    14,4 & 28 &       &    \\
    14,5 & 28 &       &    \\
    \bottomrule
  \end{tabular}
\end{table}

\begin{table}
  \centering
  \caption{Messwerte für Strontium und Zink.}
  \label{tab:strontiumZink}
  \begin{tabular}{c c | c c}
    \toprule
    $\theta_{\symup{Strontium}}^{\circ}$ & $N_{\symup{Strontium}}/Imp/s$ & $\theta_{\symup{Zink}}^{\circ}$ & $N_{\symup{Zink}}/Imp/s$ \\
    \midrule
    10,0 &  46 & 18,0 &  61 \\
    10,1 &  46 & 18,1 &  59 \\
    10,2 &  41 & 18,2 &  59 \\
    10,3 &  46 & 18,3 &  69 \\
    10,4 &  43 & 18,4 &  81 \\
    10,5 &  43 & 18,5 &  86 \\
    10,6 &  46 & 18,6 & 101 \\
    10,7 &  63 & 18,7 & 103 \\
    10,8 &  92 & 18,8 & 113 \\
    10,9 & 117 & 18,9 & 104 \\
    11,0 & 158 & 19,0 & 102 \\
    11,1 & 178 & 19,1 & 106 \\
    11,2 & 194 & 19,2 & 100 \\
    11,3 & 194 &      &     \\
    11,4 & 195 &      &     \\
    11,5 & 184 &      &     \\
    11,6 & 185 &      &     \\
    11,7 & 170 &      &     \\
    11,8 & 170 &      &     \\
    11,9 & 164 &      &     \\
    12,0 & 160 &      &     \\
    \bottomrule
  \end{tabular}
\end{table}

\begin{table}
  \centering
  \caption{Messwerte für Zirkonium}
  \label{tab:zirkonium}
  \begin{tabular}{c c}
    \toprule
    $\theta_{\symup{Zirkonium}}^{\circ}$ & $N_{\symup{Zirkonium}}/Imp/s$ \\
    \midrule
     8,8 & 119 \\
     8,9 & 117 \\
     9,0 & 114 \\
     9,1 & 113 \\
     9,2 & 109 \\
     9,3 & 113 \\
     9,4 & 107 \\
     9,5 & 104 \\
     9,6 & 110 \\
     9,7 & 132 \\
     9,8 & 175 \\
     9,9 & 198 \\
    10,0 & 247 \\
    10,1 & 271 \\
    10,2 & 273 \\
    10,3 & 278 \\
    10,4 & 280 \\
    10,5 & 270 \\
    10,6 & 279 \\
    10,7 & 279 \\
    10,8 & 276 \\
    10,9 & 273 \\
    11,0 & 273 \\
    \bottomrule
  \end{tabular}
\end{table}

\begin{figure}
  \centering
  \includegraphics{build/brom.pdf}
  \caption{Messung des Bromabsorbers.}
  \label{fig:brom}
\end{figure}

\begin{figure}
  \centering
  \includegraphics{build/gallium.pdf}
  \caption{Messung des Galliumabsorbers.}
  \label{fig:gallium}
\end{figure}

\begin{figure}
  \centering
  \includegraphics{build/strontium.pdf}
  \caption{Messung des Strontiumabsorbers.}
  \label{fig:strontium}
\end{figure}

\begin{figure}
  \centering
  \includegraphics{build/zink.pdf}
  \caption{Messung des Zinkabsorbers.}
  \label{fig:zink}
\end{figure}

\begin{figure}
  \centering
  \includegraphics{build/zirkonium.pdf}
  \caption{Messung des Zirkoniumabsorbers.}
  \label{fig:zirkonium}
\end{figure}

Die Ergebniss für die Energien und Abschirmkonstanten und deren Abweichung vom Theoriewert sind in \autoref{tab:E}
zu finden.
\begin{table}
  \caption{Energien und Abschirmkonstanten.}
  \label{tab:E}
  \begin{tabular}{c | c c c c}
    \toprule
    Element & $E/\unit{\kilo\eV}$ & Abweichung Theorie & $\sigma$ & Abweichung Theorie \\
    \midrule
    Br & 13,283 & $1,41\%$ & 4,07 & $5,71\%$ \\
    Ga & 10,181 & $1,86\%$ & 3,87 & $7,20\%$ \\
    Sr & 15,849 & $1,58\%$ & 4,27 & $6,75\%$ \\
    Zn &  9,651 & $0,01\%$ & 3,56 & $0\%$ \\
    Zr & 17,554 & $2,48\%$ & 4,55 & $10,98\%$ \\
    \bottomrule
  \end{tabular}
\end{table}

\subsection{Moseleysches Gesetz}
\label{sec:MoseleyschesGesetz}
Nachdem Moseleyschen Gesetz ist $E_{\symup{K}}$ proportional zu $Z^2$. Durch eine lineare Regression kann so
die Rydbergenergie bestimmt wird. Dazu wird die $\sqrt{E_{\symup{K}}}$ gegen $Z$ in \autoref{fig:Rydberg} 
aufgetragen.
\begin{figure}
  \centering
  \includegraphics{build/rydberg.pdf}
  \caption{Lineare Regression um die Rydbergenergie zu bestimmen.}
  \label{fig:Rydberg}
\end{figure}
Die Parameter der linearen Regression ergeben sich zu
\begin{align*}
  a &= (0,300 \pm 0,011)\,\frac{1}{\sqrt{\unit{\eV}}} \\
  b &= (0,233 \pm 1,283).
\end{align*}
Für die Rydbergenergie folgt dann
\begin{align*}
  R_{\infty}\cdot h &= \frac{1}{a^2} \\
  R_{\infty} &= (9,01 \pm 0.07)\,\unit{\eV}
\end{align*}
Dadurch ergibt sich eine Abweichung zum Theoriewert von $33,75\,\unit{\eV}$.