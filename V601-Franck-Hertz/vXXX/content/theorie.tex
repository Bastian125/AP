\section{Zielsetzung}
\label{sec:Zielsetzung}
Es soll die Quantennatur der Elektronenhülle von Atomen gezeigt werden. Weiterhin soll ein Zusammenhang zwischen
der Anregungsenergie und der Wellenlänge des emittierten Lichts hergestellt werden. Mit den daruas erhaltenen
Erkentnissen lassen sich die Bohrschen Postulate teilweise bestätigen. Außerdem wird die Energieverteilung der
Elektronen bestimmt.

\section{Theorie}
\label{sec:Theorie}
Es wird Hg-Dampf mit passender Dichte mit möglichst monoenergetischen Elektronen beschossen. Dabei treten
elastische und unelastische Stöße auf. Aus der Energiedifferenz der Elektronen vor und nach dem Stoß lässt
sich dann die vom Hg-Atom aufgenommene Energie bestimmen. Die unelastischen Stöße werden dabei verwendet, um
die Atome aus dem Grundzustand $E_{\symup{0}}$ in den ersten angeregten Zustand $E_{\symup{1}}$ zu heben.
Für die Energiedifferenz gilt dann
\begin{equation}
    \label{eqn:Ediff}
    \frac{m_{\symup{0}}v_{\symup{vor}}^2}{2} - \frac{m_{\symup{0}}v_{\symup{nach}}^2}{2}
    = E_{\symup{1}} - E_{\symup{0}}\,.
\end{equation}
Die Energien lassen sich mit der Gegenfeldmethode bestimmen.



\cite{sample}
