\section{Diskussion}
\label{sec:Diskussion}

Da alle Messwerte nach Augenmaß abgelesen werden, sind bei den abgelesenen Messwerten
bereits ein Messfehler vorliegend. Dieser lässt sich allerdings nicht genau bestimmen.\\
Es ist jedoch sicher anzunehmen, dass dadurch das Kotaktpotential einen unbekannt großen
Messfehler beinhaltet.\\ \\
Anhand der Messergebnisse der aufgenommenen Kurve bei $160 \, \si{\celsius}$, welche 
stetig fallen, lässt sich
die Theorie der vermehrt vorkommenden elastischen Stöße bestätigen. Zudem lässt sich dadurch,
dass die Kurve nach 4,9 V nicht mehr ansteigt, die berechnete Anregungsenergie zur Genüge
bestätigen.\\ \\
Mittels den zwei Franck-Hertz-Kurven konnten unterschiedliche Werte für die Anregungsenergie
bestimmt werden. Der erste Wert ist $\bar{E_1}&=(5,0 \pm 0,4) \, \si{\electronvolt}$ und der
zweite ist $\bar{E_2}&=(4,8 \pm 0,13) \, \si{\electronvolt}$.\\
Diese zwei Werte besitzen jeweils eine Abweichung von 4\% zueinander und eine Abweichung von
jeweils 2\% zum Literaturwert $E_{\mathrm{lit}} = 4,9 \si{\electronvolt}$\cite{lit}.\\
Da die Messunsicherheiten und Abweichungen hinreichend klein sind, lässt sich der Literaturwert
verifizieren.\\
Neben der Anregungsenergie wurde zudem die Wellenlängen für die Übergänge der Grundzustände 
berechnet. Die erste berechnete Wellenlänge ist $\lambda_1&=(248 \pm 18) \, \si{\nano\meter}$
und die zweite Wellenlänge ist $\lambda_2&=(259 \pm 7) \, \si{\nano\meter}$.\\
Die zwei Werte haben eine Abweichung von 4,44\% zueinander und eine Abweichung von 2,37\% und
1,98\% zu dem Literaturwert $\lambda_{\mathrm{lit}} = 253 \si{\nano\meter}$\cite{lit}.\\
Da die Abweichungen und Messunsicherheiten abermals gering genug sind, lässt sich auch
dieser Literaturwert bestätigen.\\