\section{Durchführung}
\label{sec:Durchführung}

\subsection{Lange Spule}
\label{sec:langeSpule}
Eine lange Spule wird an ein Netzgerät angeschlossen. Durch Einstellen des Stroms und der Spannung
wird ein Magnetfeld erzeugt, das durch eine Hallsonde gemessen wird. Dabei werden die Messungen immer
in unterschiedlichen Abständen zum Spulenmittelpunkt in der Spule gemessen.

\subsection{Helmholtz-Spulenpaar}
\label{sec:helmholtz}
Es wird ein Helmholtz-Spulenpaar an ein Netzgerät angeschlossen. Mit Hilfe einer transversalen Hallsonde
wird die Feldstärke innerhalb und außerhalb der Spule gemessen. Im Inneren werden jeweils die Abstände
zum Spulenmittelpunkt variiert.

\subsection{Ringspule}
\label{sec:ringspule}
Es wird eine Helmholtz-Spule mit Eisenkern in der Mitte verwendet. Der Strom wird
auf $10$ Ampere erhöht. Danach wird er wieder auf $0$ Ampere runter gefahren. Die Strom wird umgepolt
und wieder bis auf $10$ Ampere erhöht. Danach wird der Strom wieder auf $0$ Ampere runter gefahren, umgepolt
und bis auf $10$ Ampere erhöht. Dies geschieht jeweils in $1$ Ampere-Schritten und währenddessen wird die
magnetische Feldstärke mit Hilfe einer Hallsonde gemessen.