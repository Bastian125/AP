\section{Zielsetzung}
\label{sec:Zielsetzung}
Ziel des Versuchs ist es die Feldstärken unterschiedlicher Spulen in verschiedenen Anordnungen in
Abhängigkeit der Ortes zu messen. Weiterhin wird mit einer Ringspule ein Eisenkern magnetisiert und eine
Hystereskurve dazu aufgestellt.

\section{Theorie}
\label{sec:Theorie}
Im Allgemeinen erzeugen bewegte Ladungen magnetische Felder. Die magnetische Feldstärke ist eine vektorielle Größe
und ist durch
\begin{equation}
    \label{eqn:magFeldstaerke}
    \vec{H} = \mu_{0}\cdot\mu_{\symup{r}}\cdot \vec{B}
\end{equation}
gegeben. Dabei ist $\mu_{0}$ die Vakuumpermeabilität, $\mu_{\symup{r}}$ die materialabhängige relative
Permeabilität und $\vec{B}$ die magnetische Flussdichte. Die beiden Permeabilitäten lassen sich zu der gesamt
Permeabilität $\mu = \mu_0 \cdot \mu_{\symup{r}}$ zusammenfassen. Um die Magnetfeldstärken beliebiger
Leiterschleifen zu bestimmen, wird das Biot-Savart-Gesetz verwendet, das durch
\begin{equation}
    \label{eqn:biotsavart}
    \vec{B}(r) = \frac{\mu_{0}I}{4\pi} \int_{\Gamma} \frac{\vec{ds}\times \vec{r}}{r^3}
\end{equation}
bestimmt ist. $I$ ist in diesem Fall die Stromstärke und $\Gamma$ der Weg der Schleife. Für eine Spule mit
$n$ Windungen ergibt sich dann
\begin{equation}
    \label{eqn:nspule}
    \vec{B}(x) = \frac{n \mu_0 I}{2} \frac{R^2}{(R^2 + x^2)^{\frac{3}{2}}} \cdot \hat{x}.
\end{equation}
$R$ ist dabei der Radius der Spule und $x$ ist der Abstand zum Spulenzentrum.
\cite{sample}
