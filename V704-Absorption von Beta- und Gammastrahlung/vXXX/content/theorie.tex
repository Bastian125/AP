\section{Zielsetzung}
\label{sec:Zielsetzung}
In diesem Versuch soll die Wechselwirkung von $\gamma$- und $\beta^{-}$-Strahlung mit Materie untersucht werden.

\section{Theorie}
\label{sec:Theorie}
So bald $\beta$- oder $\gamma$-Strahlen auf Materie, die hier als Absorber bezeichnet werden, treffen, finden
Wechselwirkungen statt, die zu einer Abnahme der Intensität führen. Ein Maß für die Wahrscheinlichkeit, dass
ein Teilchen mit einem Absorber wechselwirkt ist der Wirkungsquerschnitt $\sigma$. Der Wirkungsquerschnitt lässt
sich als Fläche der Zielscheibe, auf die eine Teilchen trifft, vorstellen. Dadurch kann die Anzahl der
Wechselwirkungen durch
\begin{align*}
    N = N_{\symup{0}} n D \sigma
\end{align*}
bestimmt werden. Dabei ist $D$ die Dicke des Absorbers $n$ die Anzahl der Teilchen pro Volumeneinheit und
$N_{\symup{0}}$ die Anzahl der Teilchen, die pro Zeiteinheit auf das Material treffen. Durch Betrachtung eines
infitisemalen dicken Absorbers und Umstellen ergibt sich eine negative Exponentialfunktion für die Anzahl der
Wechselwirkungen
\begin{align}
    \label{eqn:Anzahl}
    N(D) = N_{\symup{0}} e^{-n \sigma D}.
\end{align}
$n \sigma$ wird meist durch den Absorbtionskoeffizienten
\begin{align}
    \label{eqn:Absorbtionskoeffizient}
    \mu = n\sigma
\end{align}
ausgedrückt. $n$ bestimmt sich in \autoref{eqn:Anzahl} durch
\begin{align*}
    n = \frac{zN_{\symup{L}}}{V_{\symup{Mol}}} = \frac{zN_{\symup{L}}\rho}{M}
\end{align*}
ausdrücken. Dabei ist $z$ die Ordnungszahl, $N_{\symup{L}}$ die Loschmidtsche Zahl, $V_{\symup{Mol}}$ das
Molvolumen, $M$ das Molekulargewicht und $\rho$ die Dichte.

\subsection{Gammastrahlung}
\label{sec:Gammastrahlung}
Gammastrahlung wird emmitiert, wenn Elektronen durch Quantensprünge aus energetisch angeregten Zuständen
in energetisch niedrigere Zustände fallen. Die Energiedifferenz wird dann als $\gamma$-Strahlung durch Photon
mit der Energie
\begin{align*}
    E = hf
\end{align*}
emmitiert. Trifft die Strahlung nun auf Materie sind die in Abbildung 1 dargestellten
Wechselwirkungsprozesse möglich.
\begin{figure}
    \centering
    \label{fig:WWProzesse}
    \includegraphics[width=\textwidth]{Bilder/Wechselwirkung.png}
    \caption{Wechselwirkungsprozesse der $\gamma$-Strahlung mit Materie \cite{sample}.}
\end{figure}
Die drei dominierenden Effekte sind hier Photo- und Compton-Effekt und die Paarbildung.

\subsubsection{Photoeffekt}
\label{sec:Photoeffekt}
Trifft das $\gamma$-Quant auf einen Hüllenelektron, so wird das Elektron herausgelöst und das Photon vernichtet.
Es erhält dabei die Energie
\begin{align*}
    E_{\symup{e}} = hf - E_{\symup{B}}.
\end{align*}
Dabei ist $E_{\symup{B}}$ die Bindungsenergie des Elektrons. Damit der Photoeffekt auftritt, muss das Photon
mindestens die Bindungsenergie des Elektrons haben.

\subsubsection{Compton-Effekt}
\label{sec:Compton-Effekt}
Beim Compton-Effekt trifft das Photon auf ein freies Elektron, wie es zum Beispiel bei den leitenden Elektronen
von Metallen zu finden ist. Dabei wird das Photon inelastisch gestreut und gibt einen Teil seiner Energie an das
Elektron ab. Dieser Prozess ist in Abbildung 2 dargestellt.
\begin{figure}
    \centering
    \label{fig:compton}
    \includegraphics[width=\textwidth]{Bilder/Compton.png}
    \caption{Schematische Darstellung des Comptoneffekts \cite{sample}.}
\end{figure}

\subsubsection{Paarerzeugung}
\label{sec:Paarerzeugung}
Wenn die Energie des $\gamma$-Quants größer als die doppelte Ruhemasse ist, also
\begin{align*}
    E_{\symup{\gamma}} > 2m_{\symup{0}}c^2 = 1,02\,\unit{\mega\eV}
\end{align*}
gilt, dann kann das Photon durch Erzeugung eines Elektrons und eines Positrons annihiliert werden.\\
\\
In Abbildung 3 ist der Extinsionskoeffizient gegen die Energie aufgetragen und die Anteile von Photo- und
Compton-Effekt und der Paarerzeugung aufgeschlüsselt.
\begin{figure}
    \centering
    \label{fig:Extinsionskoeffizient}
    \includegraphics[scale=0.75]{Bilder/Energieabhaengigkeit.png}
    \caption{Energieabhängigkeit des Extinsionskoeffizienten \cite{sample}.}
\end{figure}
Der Photoeffekt ist vorallem bei niedrigen Energien von Relevanz, da er mit der Energie linear abfällt.
Bei mittleren Energien dominiert hauptsächlich der Compton-Effekt bis bei ca. $1\,\unit{\mega\eV}$ die
Paarbildung beginnt und ansteigt.

\subsection{Betastrahlung}
\label{sec:Betastrahlung}
Bei der $\beta$-Strahlung handelt es sich hauptsächlich um Elektronen bei $\beta^{-}$- und Positronen bei
$\beta^{+}$-Strahlung. Diese entsteht, wenn ein Neutron in ein Proton oder ein Proton in ein Neutron umgewandelt
wird. Weiterhin wird dabei noch ein Antineutrino oder eine Neutrino aufgrund der Leptonenerhaltung emmitiert.
\begin{align*}
    n &\rightarrow p + e^{-} + \bar{\nu_{\symup{e}}}\\
    p &\rightarrow n + e^{+} + \nu_{\symup{e}}
\end{align*}
Da die Wechselwirkung eines Neutrinos mit Materie verschwinden gering ist, wird es hier vernachlässigt.\\
\\
Obwohl eine Vielzahl von Prozessen auftreten, wird die Wechselwirkung hauptsächlich nur von den folgenden
drei Prozessen bestimmt.

\subsubsection{Elastische Streuung am Atomkern}
\label{sec:elastischeStreuung}
