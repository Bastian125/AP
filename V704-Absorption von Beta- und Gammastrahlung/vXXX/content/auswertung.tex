\section{Auswertung}
\label{sec:Auswertung}

Im Folgenden werden die aufgenommenen Messwerte ausgewertet.

\subsection{Die $γ$-Absorption}

In diesen Versuchsteil werden mithilfe der Messdaten die Absorptionskoeffizienten von Eisen (Fe) und Blei (Pb)
und die Größen N(0) bestimmt. Als $\gamma$-Strahler wird $^{137}$Cs verwendet.\\
Die Ungenauigeit der gemessenen Zerfälle bestimmt sich zu $\Delta Z = \sqrt{Z}$, da die Anzahl der Zerfälle
statistisch nach der Poissonverteilung verteilt sind.\\
Bei einer Nullmessung über 900s werden $Z_u = 567 \pm 24$ Zerfälle vom Zählrohr gemessen. Das entspricht einer
Impulsrate von 
\begin{align*}
  N_u = 0,63 \pm 0,03 \si{\per\second}.
\end{align*}
Die Ungenauigeit wurde mit der Gaußschen Fehlerfortpflanzung bestimmt.\\
Es werden nun verschieden dicke Eisen- und Bleiplatten zwischen Zählrohr und Strahler geschoben und Messungen 
durchgeführt. Die Ergebnisse lassen sich in \autoref{tab:1aFe} für Eisen und in \autoref{tab:1aPb} für Blei finden.\\
\begin{table}
  \centering
  \caption{Messergebnisse mit unterschiedlichen Eisenplatten.}
  \label{tab:1aFe}
  \begin{tabular}{c c c c}
    $d$ in \si{\milli\meter} & Messdauer $t$ in $\si{\second}$ & Anzahl Zerfälle $Z$  &  Impulsrate $N-N_u$ in $\si{\per\second}$\\
       \midrule
        $1,58 \pm 0,02$   &  300   &  $33377 \pm 183$ & $110,63 \pm 3,37$ \\
        $4,11 \pm 0,02$   &  300   &  $30768 \pm 175$ & $101,93 \pm 3,11$\\
        $6,50 \pm 0,02$   &  300   &  $27037 \pm 164$ & $89,49 \pm 2,74$\\
        $8,20 \pm 0,02$   &  300   &  $24383 \pm 156$ & $80,65 \pm 2,48$\\
        $10,92 \pm 0,02$  &  300   &  $21985 \pm 148$ & $72,65 \pm 2,24$\\
        $15,90 \pm 0,02$  &  300   &  $18337 \pm 135$ & $60,49 \pm 1,88$\\
        $18,20 \pm 0,02$  &  300   &  $15931 \pm 126$ & $52,47 \pm 1,64$\\
        $22,56 \pm 0,02$  &  300   &  $10510 \pm 103$ & $34,40 \pm 1,11$\\
        $39,81 \pm 0,02$  &  300   &  $ 6657 \pm 82$ & $21,56 \pm 0,74$\\
        $50,00 \pm 0,02$  &  300   &  $ 4557 \pm 68$ & $14,56 \pm 0,54$\\
      \bottomrule
    \end{tabular}
\end{table}

\begin{table}
  \centering
  \caption{Messergebnisse mit unterschiedlichen Bleiplatten.}
  \label{tab:1aPb}
  \begin{tabular}{c c c c}
    $d$ in \si{\milli\meter} & Messdauer $t$ in $\si{\second}$ & Anzahl Zerfälle $Z$  &  korrigierte Impulsrate $N-N_u$ in $\si{\per\second}$\\
       \midrule
        $3,79 \pm 0,02$   &  300   &  $24245 \pm 156$ & $80,19 \pm 2,47$ \\
        $5,20 \pm 0,02$   &  300   &  $22575 \pm 150$ & $74,52 \pm 2,29$\\
        $9,40 \pm 0,02$   &  200   &  $11221 \pm 106$ & $55,48 \pm 1,74$\\
        $14,30 \pm 0,02$  &  200   &  $5936  \pm 77$ & $29,05 \pm 0,97$\\
        $20,00 \pm 0,02$  &  200   &  $3926  \pm 63$ & $19,00 \pm 0,69$\\
        $26,18 \pm 0,02$  &  200   &  $2067  \pm 45$ & $9,71 \pm 0,43$\\
        $46,10 \pm 0,02$  &  200   &  $602   \pm 25$ & $0,37 \pm 0,24$\\
      \bottomrule
    \end{tabular}
\end{table}

Es werden für Eisen zehn verschiedene Dicken an Platten verwendet, bei Blei sieben. Es werden zudem 
die um den Nulleffekt korrigierten Impulsraten $N-N_u$ berechnet und eingetragen.\\
Nachfolgend werden die Absorptionskoeffizienten $ \mu $ bestimmt. Dazu wird die \autoref{eqn:Absorptionsgesetz}
verwendet. Der Logarithmus der korrigierten Impulsraten wird gegen die Schichtdicke $d$ aufgetragen, wodurch sich
ein linearer Zusammnehang ergibt. Die lineare Regression ergibt für die Geraden folgende Geradengleichungen:
\begin{align*}
  f_{Fe}(d) = (-42,7 \pm 2,0)d + (4,7 \pm 0,1)\\
  f_{Pb}(d) = (-85,6 \pm 3,5)d + (4,7 \pm 0,1)
\end{align*}

Die Fehlerfortpflanzung wurde mittels Python bestimmt.

\begin{figure}
  \centering
  \includegraphics{Bilder/Eisen.pdf}
  \caption{Logarithmus der korrigierten Impulsrate bei Eisen aufgetragen gegen die Schichtdicke $d$}
  \label{fig:Eisen}
\end{figure}
\begin{figure}
  \centering
  \includegraphics{Bilder/Blei.pdf}
  \caption{Logarithmus der korrigierten Impulsrate bei Blei aufgetragen gegen die Schichtdicke $d$}
  \label{fig:Blei}
\end{figure}

Durch die Geradensteigungen und y-Achsenabschnitte der Geraden in \autoref{fig:Eisen} und \autoref{fig:Blei} 
lassen sich nun nach \autoref{eqn:Absorptionsgesetz} die Absorptionskoeffizienten $\mu_{Fe}$ und $\mu_{Pb}$ 
bestimmen, sowie $N_0$ der beiden Materialien:\\
\begin{align*}
  \mu_{Fe} &= (-42,7 \pm 2,0) \, \si{\per\meter} \\
  N_{0,Fe} &= (110 \pm 1) \, \si{\per\second} \\
  \mu_{Pb} &= (-85,6 \pm 3,5) \, \si{\per\meter} \\
  N_{0,Pb} &= (110 \pm 1) \, \si{\per\second} \\
\end{align*}
Dadurch lassen sich nun die Absorptionsgesetze bestimmen:
\begin{align*}
  N_{Fe}(d) = (110 \pm 1) \, \si{\per\second} \cdot e^{(-42,7 \pm 2,0) \, (\si{\per\meter}) d} \\
  N_{Pb}(d) = (110 \pm 1) \, \si{\per\second} \cdot e^{(-85,6 \pm 3,5) \, (\si{\per\meter}) d} \\
\end{align*}

Die Fehlerfortpflanzungen der Gleichungen wurden allesamt mittels Python bestimmt.

\subsection{Vergleich mit den aus der Theorie berechneten Absorptionskoeffizienten}

Um Schlüsse über die vorliegenden Absorbtionsmechanismen zu ziehen, werden die gemessenen Absorptionskoeffizienten
mit gerechneten Werten verglichen. Die Absorptionskoeffizienten $\mu_{Com}$ werden durch \autoref{eq:sigma_c} und
\autoref{eq:mu_c} berechnet.
Für diese Gleichungen werden einige Materialkonstanten benötigt, diese werden in \autoref{tab:Material} notiert.
\begin{table}
  \centering
  \begin{tabular}{S S[table-format=2.0] S[table-format=3.1] S[table-format=2.3]}
    \toprule
    {Material} & {Ordnungszahl $Z$} & {Masse $m\:$ in $\frac{\si\gram}{\si\mol}$} & {Dichte $\rho\:$ in $\frac{\si\gram}{\si{\centi\meter}^3}$}\\
    \midrule
    \text{Blei} & 82 & 207.2 & 11.342\\
    \text{Eisen} &26 & 55.8  & 7.874\\
    \bottomrule
  \end{tabular}
  \caption{Materialkonstanten.}
  \label{tab:Material}
\end{table}
Es wird zudem die charakteristische Größe $\epsilon = 1,295$ für Caesium verwendet.
Einsetzen ergibt nun:
\begin{equation*}
  \sigma_{Com} = 2,57 \cdot 10^{-29} \, \mathrm{m}^2
\end{equation*}
und dadurch die Absorbtionskoeffizienten
\begin{align*}
  \mu_{Com,Fe} = 56,76 \, \si{\per\meter} \, \mathrm{und} \\
  \mu_{Com,Pb} = 69,43 \, \si{\per\meter}. \\
\end{align*}

\subsection{\texorpdfstring{$\beta^-$}{Beta}-Absorption}

Um die Maximalenergie von $^{99}$Tc zu bestimmen, wird mithilfe der aufgenommenen Messwerte eine
$\beta$-Absorptionskurve erstellt.\\
In \autoref{tab:Aluminium} sind die Messwerte zu verschiedenen Dicken der Aluminiumplatten eingetragen,
sowie die berechneten korrigierten Impulsraten $N - N_u$.
\begin{table}
  \centering
  \caption{Messergebnisse mit unterschiedlichen Aluminiumplatten.}
  \label{tab:Alluminium}
  \begin{tabular}{c c c c}
    $d$ in \si{\micro\meter} & Messdauer $t$ in $\si{\second}$ & Anzahl Zerfälle $Z$  &  korrigierte Impulsrate $N-N_u$ in $\si{\per\second}$\\
       \midrule
       $100 \pm 0,5$ &    500  &   $14164 \pm 119$  & $27,70 \pm 0,86$ \\
       $125 \pm 1$ &   500   &   $3514 \pm 59$  &     $6,40 \pm 0,22$\\
       $153 \pm 1$  & 600   &   $4317 \pm 66$ &       $6,57 \pm 0,23$\\
       $160 \pm 1$  &  600  &    $2595 \pm 51$  &     $3,70 \pm 0,14$\\
       $200 \pm 1$  & 700    &  $1199 \pm 35$ &       $1,08 \pm 0,06$\\
       $253 \pm 1$  &  700   &    $523 \pm 23$  &     $0,12 \pm 0,03$\\
       $302 \pm 1$  & 700    &  $ 506 \pm 22$ &       $0,09 \pm 0,03$\\
       $400 \pm 1$  & 700   &    $466 \pm 22$ &       $0,04 \pm 0,03$\\
       $482 \pm 1$  &  700  &    $ 495 \pm 20$  &     $0,08 \pm 0,03$\\
      \bottomrule
    \end{tabular}
\end{table}
Die korrigierte Impulsrate wird nun logarithmisch gegen die Schichtdicke $d$ aufgetragen, wodurch die
Massenbelegung $R_{max}$ und die Gesamtenergie $E_{max}$ berechnet werden können. Die maximale Reichweite
ist der Schnittpunkt zwischen der Ausgleichsgeraden des vorderen Bereichs der Messwerte und der Ausgleichsgeraden
des hinteren Bereichs der Messwerte.\\
Die Geradengleichungen der beiden Geraden lauten
\begin{align*}
  g(d) = (-29710 \pm 5011)d + (6.1 \pm 0.8) \, \mathrm{und}
  f(d) = (-2.6 \pm 0.3).
\end{align*}
Dadurch ergibt sich die maximale Reichweite $D = 0,000292 \pm 0,000056$ m. Mit der Massenbelegung $R_{max} = \rho D$ lässt
sich dann mit \autoref{eqn:emax} die Gesamtenergie bestimmen zu $E_{max} = (0,295 \pm 0,115)$ MeV.
\begin{figure}
  \centering
  \includegraphics{Bilder/beta.pdf}
  \caption{Logarithmus der korrigierten Impulsrate bei Alluminium aufgetragen gegen die Schichtdicke $d$}
  \label{fig:Beta}
\end{figure}