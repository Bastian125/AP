\section{Aufbau und Durchführung}
\label{sec:Durchfuehrung}
In Abbildung 5 ist der Aufbau schematisch dargestellt. Es wird ein Geiger-Müller-Zählrohr, das an ein
elektrisches Zählwerk angeschlossen ist, in einer Linie mit einer Strahlenquelle angeschlossen. Dazwischen
lässt sich ein Absorbermaterial befestigen und sowohl das Zählrohr als auch die Quelle sind mit Blei abgeschirmt.
\begin{figure}
    \centering
    \label{fig:Aufbau}
    \includegraphics{Bilder/Aufbau.png}
    \caption{Schematischer Versuchsaufbau \cite{sample}.}
\end{figure}
Zu Beginn wird für $t=900\,\unit{s}$ eine Nullmessung durchgeführt, um die Hintergrundstrahlung zu ermitteln.
Als $\gamma$-Strahler wird Cs-137 verwendet und für Aluminium mit unterschiedlichen Dicken gemessen. Für den
$\beta$-Strahler wird Tc-99 verwendet und für unterschiedliche Dicken von Eisen und Blei gemessen.