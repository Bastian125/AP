\section{Diskussion}
\label{sec:Diskussion}
\subsection{Vergleich der berechneten und gemessenen Werte von \texorpdfstring{$\mu$}{Absorptionskoeffizient}}

Die berechneten und gemessenen Werte des Absorptionskoeffizienten haben bei Eisen eine Abweichung von 32,93 \%. 
Das ist eine recht hohe Abweichung, welche durch einen systematischen Fehler der Messapparatur
entstanden sein könnte.\\
Bei Blei haben gemessener und berechneter Wert eine Abweichung von rund 23,29 \%. Da beide Abweichungen so hoch sind,
ist die Annahme des systematischen Fehlers bestätigt.\\
Da die Abweichungen, obwohl sie etwas hoch sind, dennoch nicht enorm hoch sind, lässt sich das Messverfahren
als geeignet anzunehmen.\\

\subsection{Maximale Energie der \texorpdfstring{$\beta^-$}{Beta}-Strahlung}

Der Literaturwert der maximalen Energie beträgt $E_{lit} = 0,294 $ MeV. Somit ergibt sich eine Abweichung von nur
0,34 \%. Dadurch erweist sich dieses Messverfahren als äußerst genau.\\
Da für die Messungen mit dem $\beta$-Strahler ein anderer Versuchsaufbau verwendet wurde, lässt sich der systematische
Fehler des Versuchsaufbaus der anderen Versuchsreihe nicht widerlegen.