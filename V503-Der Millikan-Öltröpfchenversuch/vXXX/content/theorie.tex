\section{Zielsetzung}
\label{sec:Zielsetzung}

In diesem Versuch soll unter Ausnutzung eines elektrischen Feldes und der Gravitation
die Elementarladung $e_0$ bestimmt werden.

\section{Theorie}
\label{sec:Theorie}

Beim Zerstäuben von Öltröpchen werden diese durch die gegenseitige Reibung aneinander
elektrisch geladen. Wenn diese geladenen Tröpchen nun in ein elektrisches Feld $\vec{E}$ gebracht
werden, wirkt eine elektrische Kraft entlang des Feldes auf das Tröpchen.\\
Um ein möglichst homogenes Feld zu erzeugen, kann man einen Plattenkondensator verwenden.
Je nach Ladung $q$ des Tröpchens, verspürt dieses zu der positiv oder negativ geladenen Platte 
eine Kraft. Ein positiv geladenes Tröpchen wird zu der negativ geladenen Platte hin
beschleunigt und ein negativ geladenes zu der positiv geladenen Platte.\\
Da das Tröpchen eine Masse $m$ besitzt, wirkt neben einer elektrischen Kraft 
$\vec{F}_{\mathrm{el}} = q \cdot \vec{E}$ auch die Gravitationskraft $\vec{F}_{\mathrm{g}} = m \vec{g}$
auf dieses. Dadurch wird das Teilchen nach unten beschleunigt. Durch diese Fallbewegung
entstehen Stöße mit den Luftmolekülen und das Tröpchen wird durch die Stokesche
Reibungskraft 
\begin{equation}
    \label{eqn:Stokes}
    \vec{F}_{\mathrm{R}} = - 6 \pi r \mu_{\mathrm{L}} \vec{v}
\end{equation}
entgegen der Gravitationskraft beschleunigt und somit abgebremst. $r$ ist hier der 
Radius des Tröpchens und $\mu_{\mathrm{L}}$ die Viskosität der Luft.\\
Nach einer Zeit ist die Reibung genauso stark wie die Beschleunigung nach unten und es 
stellt sich eine Gleichgewichtsgeschwindigkeit $\vec{v}_0$ ein. Die Kräftegleichung
ohne elektrisches Feld ergibt sich zu 
\begin{equation*}
    \frac{4 \pi}{3} r^3 (\rho_{\mathrm{Öl}} - \rho_{\mathrm{L}}) g = 6 \pi \mu_{\mathrm{L}} r v_0.
\end{equation*}
Nach Umstellen ergibt sich somit der Radius des Öltröpchens zu 
\begin{equation*}
    r = \sqrt{\frac{9 \mu_{\mathrm{L}} v_0}{2 g (\rho_{\mathrm{Öl}} - \rho_{\mathrm{L}})}}.
\end{equation*}
Wenn man die geladenen Tröpchen nun in das E-Feld bringt, wirkt eine elektrische Kraft auf diese,
wodurch das Teilchen entweder nach oben oder unten beschleunigt wird und dieses sich dadurch
nicht mehr mit einer Geschwindigkeit $v_0$ bewegt.\\
Für die neue Kräftegleichung folgt dann:
\begin{equation}
    \label{eqn:ab}
    \frac{4 \pi}{3} r^3 (\rho_{\mathrm{Öl}} - \rho_{\mathrm{L}}) g - 6 \pi \mu_{\mathrm{L}} r v_{\mathrm{ab}} = -qE.
\end{equation}
$v_{\mathrm{ab}}$ ist die resultierende Sinkgeschwindigkeit des Tröpchens.\\
Wenn das elektrische Feld nun umgepolt wird, erfährt das Tröpchen eine Auftriebskraft und es folgt
\begin{equation}
    \label{eqn:auf}
    \frac{4 \pi}{3} r^3 (\rho_{\mathrm{Öl}} + \rho_{\mathrm{L}}) g + 6 \pi \mu_{\mathrm{L}} r v_{\mathrm{auf}} = +qE.
\end{equation}

Aus \autoref{eqn:ab} und \autoref{eqn:auf} kann dann die Ladung $q$ und der Radius $r$ des Tröpchens
bestimmt werden. Die besagten Größen können durch die Gleichungen
\begin{equation}
    \label{eqn:q}
    q = 3 \pi \mu_{\mathrm{L}} \sqrt{\frac{9}{4}\frac{\mu_{\mathrm{L}}}{g}\frac{(v_{\mathrm{ab}}-v_{\mathrm{auf}})}{(\rho_{\mathrm{Öl}} - \rho_{\mathrm{L}})}}\frac{(v_{\mathrm{ab}}+v_{\mathrm{auf}})}{E}
\end{equation}
und
\begin{equation}
    \label{eqn:r}
    r = \sqrt{\frac{9}{2}\frac{\mu_{\mathrm{L}}}{g}\frac{(v_{\mathrm{ab}}-v_{\mathrm{auf}})}{(\rho_{\mathrm{Öl}} - \rho_{\mathrm{L}})}}
\end{equation}
bestimmt werden.\\
Da die Öltröpchen kleiner sind als die mittlere freie Weglänge der Luft $\bar{l}$, lässt sich das
Gesetz von Stokes nicht anwenden. Stattdessen muss bei der Ladung $q$ für die Viskosität der Luft $\mu_{\mathrm{L}}$
ein Korrekturterm verwendet werden:
\begin{equation}
    \label{eqn:Korrekturterm}
    \mu_{\mathrm{eff}} = \mu_{\mathrm{L}} \left(\frac{1}{1 + A \frac{1}{r}}\right) = \mu_{\mathrm{L}} \left(\frac{1}{1 + B \frac{1}{pr}}\right)
\end{equation}
$B$ ist eine Konstante mit dem Wert $B = 6,17 \cdot 10^{-3} \, \mathrm{Torr \cdot cm}$. Für die korrigierte
Ladung $q$ gilt dann
\begin{equation}
    \label{eqn:qKorr}
    q^{2/3} = q_0^{2/3} (1 + B / (p \cdot r)).
\end{equation}

\cite{sample}
