\section{Diskussion}
\label{sec:Diskussion}
In \autoref{tab:fazit} sind die Elementarladung in unkorrigierter und korrigierter Form und die Avogadrokonstante
angegeben. Weiterhin sind noch die entsprechenden Literaturwerte und deren Abweichungen gegeben. Der Literaturwert
für die Elementarladung beträgt $e=1,602\cdot 10^{-19}$ \cite{e} und der der Avogadrokonstante $N_{\symup{A}}=
6,022\cdot 10^{23}$ \cite{avogadro}.
\begin{table}
    \centering
    \caption{Experimentell bestimmte Größen, deren Literaturwerte und Abweichungen zum Literaturwert.}
    \label{tab:fazit}
    \begin{tabular}{c | c c c}
        \toprule
        Größe & Experimentell & Literaturwert & Relative Abweichung \\
        \midrule
        $e_{\symup{0}}$         & $(1,1 \pm 1,4)\cdot 10^{-19}\,\si{\coulomb}$  & $1,602\cdot 10^{-19}\,\si{\coulomb}$  & $45,64\,\%$ \\
        $e_{\symup{0, korr}}$   & $(1,3 \pm 1,2)\cdot 10^{-19}\,\si{\coulomb}$  & $1,602\cdot 10^{-19}\,\si{\coulomb}$  & $23,23\,\%$ \\
        $N_{\symup{A}}$         & $(8 \pm 7)\cdot 10^{23}\,\frac{1}{\si{\mol}}$ & $6,022\cdot 10^{23}$                  & $32,85\,\%$ \\ 
        \bottomrule
    \end{tabular}
\end{table}
Insgesamt lässt sich feststellen, dass die Korrektur nach Cunningham für eine wesentlich kleinere relative
Abweichung zum Theoriewert für die Elementarladung sorgt. Weiterhin lässt sich feststellen, dass die bestimmten
Ladungen der Tröpfchen große Unsicherheiten aufweisen, die sich durch verschiedene systematische Fehler erklären
lassen. So ist zu bemerken, dass alle Zeiten durch eine Stoppuhr, die nicht von der Person, die durch die Kamera
gesehen hat, aufgenommen wurde, so dass die Reaktionszeit zweier Personen mit einbezogen werden muss. Außerdem
haben die Tröpfchen im E-Feld nicht nur eine vertikale Komponente der Geschwindigkeit, sondern bewegen sich auch
horizontal fort. Einige Tröpfchen sind nach kurzer Zeit nicht mehr im Fokus der Apparatur oder Stoßen mit anderen
Tröpfchen, so dass sich ihre Geschwindigkeit ändert.
Da die Avogadrokonstante aus $e_{\symup{0, korr}}$ bestimmt wird gilt die gleiche Argumentation für die
relative Abweichung zum Theoriewert der Avogadrokonstante. 
In Anbetracht der diskutierten Fehlerquellen sind die relativen Abweichungen verhältnismäßig gering, so dass
das Experiment zufriedenstellende Ergebniss geliefert hat.

\addsec{Anhang}
\label{sec:Anhang}

\begin{figure}
    \centering
    \includegraphics[width=0.7\textwidth]{bilder/thermistorwiderstand.png}
    \caption{Thermistor-Widerstandstabelle \cite{sample}.}
    \label{fig:thermistor}
\end{figure}

\begin{figure}
    \centering
    \includegraphics[width=0.7\textwidth]{bilder/viskositaet.png}
    \caption{Viskosität der Luft in Abhängigkeit der Temperatur \cite{sample}.}
    \label{fig:viskositaet}
\end{figure}

\begin{figure}
    \centering
    \includegraphics[width=0.7\textwidth]{bilder/Spannung1.jpg}
    \caption{Messwerte der ersten Spannung.}
\end{figure}

\begin{figure}
    \centering
    \includegraphics[width=0.7\textwidth]{bilder/Spannung2.jpg}
    \caption{Messwerte der zweiten Spannung.}
\end{figure}

\begin{figure}
    \centering
    \includegraphics[width=0.7\textwidth]{bilder/Spannung3.jpg}
    \caption{Messwerte der dritten Spannung.}
\end{figure}

\begin{figure}
    \centering
    \includegraphics[width=0.7\textwidth]{bilder/Spannung4.jpg}
    \caption{Messwerte der vierten Spannung.}
\end{figure}

\begin{figure}
    \centering
    \includegraphics[width=0.7\textwidth]{bilder/Spannung5.jpg}
    \caption{Messwerte der fünften Spannung.}
\end{figure}