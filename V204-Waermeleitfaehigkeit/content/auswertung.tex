\section{Auswertung}
\label{sec:Auswertung}
Aufgrund der Menge der Daten sind die Datensätze im Protokoll nicht mit angegeben und es wurde aus Gründen der Lesbarkeit jeder Graph mit einer durchgezogenen Linien
und keinen einzelnen Punkten markiert.
Sie können aber auf Anfrage online zugeschickt werden.\\
\\
Die Wärmeleitfähigkeit von vier unterschiedlichen Metallstäben wird einmal durch eine statische und
eine dynamische Methode bestimmt.

\subsection{Statische Methode}
\label{sec:stat}

In \autoref{fig:T1T4} sind die zeitlichen Verläufe von $T_1$, $T_4$, $T_5$ und $T_8$ aufgetragen. Diese stellen die vom Peltierelement weiter
entfernten Thermoelemente dar.

\begin{figure}[h]
  \centering
  \includegraphics{plot.pdf}
  \caption{Temperaturverlauf der Stäbe außen.}
  \label{fig:T1T4}
\end{figure}
Die Verläufe haben alle eine kurzen Startbereich bis ca. $10 \,\si{\second}$ bei dem die Verläufe nicht
ansteigen. Danach stellt sich bei allen Verläufen eine Sättigungskurve ein.
Dabei fällt auf, dass der Temperaturverlauf des Edelstahlstabs am langsamsten
ansteigt und auch insgesamt die Form der Sättigungskurve abflacht.\\
\\
Nach $700\,\si{\second}$ hat der Aluminiumstab die höchste Temperatur erreicht. Darauf folgen in absteigender Reihenfolge der breite Messingstab, der schmale Messingstab
und der Edelstahlstab. Demnach hat Aluminium die größe Wärmeleitfähigkeit von den vier Stäben.

\begin{table}[h]
  \centering
  \caption{Äußere Temperatur nach $700\,\si{\second}$.}
  \label{tab:700s}
  \begin{tabular}{c c c c c}
    \toprule
    $t$ & Messing (breit) & Messing (schmal) & Aluminium & Edelstahl \\
    \midrule
    $700\,\si{\second}$ & $47,62\,\si{\celsius}$ & $43,97\,\si{\celsius}$ & $49,15\,\si{\celsius}$ & $35,20\,\si{\celsius}$ \\
    \bottomrule
  \end{tabular}
\end{table}
Nun wird der Wärmestrom für jeweils 5 Messwerte bestimmt. Dabei ist $\kappa$ der Literaturwert der Wärmeleitfähigkeit des jeweiligen
Materials, $A$ die Querschnittsfläche und $\frac{\partial T}{\partial x}$ der Temperaturgradient, der als $\frac{\symup{\Delta} T}{\symup{\Delta} x}$
mit $\symup{\Delta}x \approx 3,00\,\si{\centi\meter}$ geschrieben werden kann.

Die verwendeten fünf Messungen für die Stäbe sind in \autoref{tab:5messungen} angegeben.
\begin{table}[h]
  \centering
  \caption{Temperaturen für fünf Messzeiten.}
  \label{tab:5messungen}
  \begin{tabular}{c c c c c}
    \toprule
    $t\,\mathbin{/}\,\si{\second}$ & Messing (breit) & Messing (schmal) & Aluminium & Edelstahl \\
    \midrule
     30\,\si{\second} & 5,68\,\si{\celsius} & 7,07\,\si{\celsius} & 5,69\,\si{\celsius} & 2,33\,\si{\celsius} \\
     60\,\si{\second} & 6,97\,\si{\celsius} & 8,43\,\si{\celsius} & 5,16\,\si{\celsius} & 8,65\,\si{\celsius} \\
    120\,\si{\second} & 5,27\,\si{\celsius} & 6,39\,\si{\celsius} & 3,27\,\si{\celsius} & 12,01\,\si{\celsius} \\
    240\,\si{\second} & 3,22\,\si{\celsius} & 4,44\,\si{\celsius} & 1,91\,\si{\celsius} & 11,44\,\si{\celsius} \\
    480\,\si{\second} & 2,28\,\si{\celsius} & 3,72\,\si{\celsius} & 1,52\,\si{\celsius} & 10,25\,\si{\celsius} \\
    \bottomrule
  \end{tabular}
\end{table}


\begin{figure}[h]
  \centering
  \includegraphics{Tempdiff.pdf}
  \caption{Temperaturdifferenz des breiten Messingstabs und des Edelstahlstabs.}
  \label{fig:Tempdiff}
\end{figure}

\subsection{Dynamische Methode}
\label{dynam}

\begin{figure}[h]
  \centering
  \includegraphics{80sMess.pdf}
  \caption{Temperaturverlauf des breiten Messingstabs.}
  \label{fig:80sMess}
\end{figure}

\begin{figure}[h]
  \centering
  \includegraphics{80sAlu.pdf}
  \caption{Temperaturverlauf des Aluminiumstabs.}
  \label{fig:80sAlu}
\end{figure}

\begin{figure}[h]
  \centering
  \includegraphics{200sEdelstahl.pdf}
  \caption{Temperaturverlauf des Edelstahlstabs.}
  \label{fig:200sEdelstahl}
\end{figure}
