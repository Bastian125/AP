\section{Zielsetzung}
\label{sec:Zielsetzung}

In diesem Versuch soll die Wärmeleitfähigkeit von Aluminium, Messing und Edelstahl nach der Angström-Methode bestimmt werden. Desweiteren soll die
Wellenlänge und Frequenz der Temperaturwelle der Stäbe nach periodischer Anregung der Temperatur untersucht werden.

\section{Theorie}
\label{sec:Theorie}

Wenn sich ein System nicht im Temperaturgleichgewicht befindet, führt dies zu einem Wärmetransport in Richtung des Temperaturgefälles. Dieser Wärmetransport
erfolgt über Konvektion, Wärmestrahlung und Wärmeleitung. Letzteres wird bei diesem Versuch genauer untersucht. Bei Festkörpern erfolgt dieser Wärmetransport über
Phononen und frei bewegliche Elektronen. Je mehr frei bewegliche Elektronen ein Körper besitzt, desto besser ist seine Wärmeleitfähigkeit. Daher haben Metalle
eine sehr hohe Wärmeleitfähigkeit. \\
Der Fluss der Wärmemenge durch die Fläche $A$ lässt sich durch die Gleichung
\begin{equation}
    \label{eqn:Waermefluss}
    \mathrm{d}Q = - \kappa A \frac{\partial T}{\partial x} \mathrm{d} t
\end{equation}
beschreiben. Die Richtung des Wärmeflusses ist dabei in Richtung der niedrigeren Temperatur. d$t$ ist die Zeitspanne, $\kappa$ ist die materialabhängige
Wärmeleitfähigkeit und $A$ ist die Querschnittsfläche, durch die die Wärme fließt. Das Material hat die Dichte $\rho$ und die spezifische Wärme $c$.\\
Die Wärmestromdichte ist durch
\begin{equation}
    \label{eqn:Waermestromdichte}
    j_w = - \kappa \frac{\partial T}{\partial x} \mathrm{ \, \,  gegeben.}
\end{equation}
Mit der Kontinuitätsgleichung kriegt man daraus die eindimensionale Wärmeleitungsgleichung:
\begin{equation}
    \label{eqn:Warmeleitungsgleichung}
    \frac{\partial T}{\partial t} = \frac{\kappa}{\rho c} \frac{\partial^2 T}{\partial x^2} .
\end{equation}
Diese Gleichung beschreibt, wie man durch die partiellen Ableitungen erkennen kann, die (eindimensionale) räumliche und zeitliche Entwicklung der Temperaturverteilung
des Körpers. Der Proportionalitätsfaktor $\frac{\kappa}{\rho c}$ ist die Wärmeleitfähigkeit $\sigma_T$ und gibt an, wie schnell sich die Temperatur im Material
ausgleicht.\\
Wenn ein langer Stab periodisch erhitzt und abgekühlt wird, pflanzt sich entlang des Temperaturgefälles eine Temperaturwelle fort. Der zeitliche und räumliche Verlauf
ist gegeben durch
\begin{equation}
    \label{eqn:Temperaturwelle}
    T(x,t) = T_{\mathrm{max}} e^{- \sqrt{\frac{\omega \rho c}{2 \kappa}}x} cos \left(\omega t - \sqrt{\frac{\omega \rho c}{2 \kappa}}x\right).
\end{equation}
die Phasengeschwindigkeit dieser Welle ist
\begin{equation}
    \label{eqn:Phasengeschwindigkeit}
    v = \frac{\omega}{k} = \frac{\omega}{\sqrt{\frac{\omega \rho c}{2 \kappa}}} = \sqrt{\frac{2 \kappa \omega}{\rho c}}.
\end{equation}
Die Dämpfung dieser Welle wird dadurch verursacht, dass sich ein Wellenberg (Höchsttemperatur) in sowohl positive als auch negative x-Richtung ausbreitet, da sich die
Wärme in Richtung des Temperaturgefälles bewegt. Den Wert der Dämpfung erhält man durch das Amplitudenverhältnis $A_{nah}$ und $A_{fern}$.\\
Desweiteren ist die Wärmeleitfähigkeit gegeben mit
\begin{equation}
    \label{eqn:Wärmeleitfähigkeit}
    \kappa = \frac{\rho c (\Delta x)^2}{2 \Delta t \ln{\left(A_{nah}/A_{fern}\right)}}.
\end{equation}
$\Delta x$ ist der Abstand der Temperaturmessstellen der Teststäbe und $\Delta t$ ist die Phasendifferenz der Temperaturwelle zwischen den Messstellen.