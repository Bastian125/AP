\section{Auswertung}
\label{sec:Auswertung}

\subsection{Vorbereitungsaufgaben}
Im Rahmen der Vorbereitungsaufgaben wird im Folgenden der Brechungsindex verschiedener Materialien recherchiert und die Gitterkonstanten verschiedener Gitter ausgerechnet.
\subsubsection{Brechungsindex verschiedener Materialien}
\begin{table}
  \centering
  \caption{Brechungsindex verschiedener Materialien.}
  \label{tab:Brechungsindex}
  \begin{tabular}{c c}
    \toprule
    Material & Brechungsindex $n$ \\
    \midrule
    Luft      & 1,0003 \cite{Wien} \\
    Wasser    & 1,333 \cite{Wien} \\
    Kronglas  & 1,510 \cite{Spektrum} \\
    Plexiglas & 1,492 \cite{Wiki} \\
    Diamant   & 2,417 \cite{Wien} \\
    \bottomrule
  \end{tabular}
\end{table}
\subsubsection{Gitterkonstanten $d$}
\begin{itemize}
  \item 600 Linien/mm; $d=\frac{10}{6} $ \textmu m
  \item 300 Linien/mm; $d=\frac{10}{3} $ \textmu m
  \item 100 Linien/mm; $d=\frac{10}{1} $ \textmu m
\end{itemize}

\subsection{Reflexionsgesetz}
Bei diesem Versuch wurde der grüne Laser, Vorlage A und ein Spiegel verwendet. Der genauere Aufbau ist bei dem Kapitel Durchführung zu finden.
Um das Reflexionsgesetz zu beweisen, wurden insgesamt 7 Messwerte für Einfalls- und Ausfallswinkel genommen. Diese sind in folgender Tabelle
aufgelistet.
\begin{table}
  \centering
  \caption{Einfalls- und Ausfallswinkel.}
  \label{tab:Aufgabe1}
  \begin{tabular}{c c c}
    \toprule
    Einfallswinkel $\alpha_1$ / $^{\circ}$& Ausfallswinkel $\alpha_2$ / $^{\circ}$ & Differenz $x$ / $^{\circ}$\\
    \midrule
    80 & 80.5 & 0.5 \\
    70 & 70.5 & 0.5\\
    60  &60.5 & 0.5\\
    50  &50 & 0\\
    40 & 40 & 0\\
    30 & 30 & 0\\
    20 & 20 & 0\\
    \bottomrule
  \end{tabular}
\end{table}
Die Winkel konnten mit einer Genauigkeit von bis zu $0,5^{\circ}$ gemessen werden. \\
Mit dieser Messungenauigkeit ergibt sich eine durchschnittliche Differenz des Einfalls- und
Ausfallswinkels von $x = 0,2143^{\circ} \pm 0,7071^{\circ}$. Die Meßunsicherheit wurde mithilfe der Gaußschen Fehlerfortpflanzung berechnet.