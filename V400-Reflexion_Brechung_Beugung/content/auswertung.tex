\section{Auswertung}
\label{sec:Auswertung}

\subsection{Vorbereitungsaufgaben}
Im Rahmen der Vorbereitungsaufgaben wird im Folgenden der Brechungsindex verschiedener Materialien recherchiert und die Gitterkonstanten verschiedener Gitter ausgerechnet.
\subsubsection{Brechungsindex verschiedener Materialien}
\begin{table}
  \centering
  \caption{Brechungsindex verschiedener Materialien.}
  \label{tab:Brechungsindex}
  \begin{tabular}{c c}
    \toprule
    Material & Brechungsindex $n$ \\
    \midrule
    Luft      & 1,0003 \cite{Wien} \\
    Wasser    & 1,333 \cite{Wien} \\
    Kronglas  & 1,510 \cite{Spektrum} \\
    Plexiglas & 1,492 \cite{Wiki} \\
    Diamant   & 2,417 \cite{Wien} \\
    \bottomrule
  \end{tabular}
\end{table}
\subsubsection{Gitterkonstanten $d$}
\begin{itemize}
  \item 600 Linien/mm; $d=\frac{10}{6} $ \textmu m
  \item 300 Linien/mm; $d=\frac{10}{3} $ \textmu m
  \item 100 Linien/mm; $d=\frac{10}{1} $ \textmu m
\end{itemize}

\subsection{Reflexionsgesetz}
Bei diesem Versuch wurde der grüne Laser, Vorlage A und ein Spiegel verwendet. Der genauere Aufbau ist bei dem Kapitel Durchführung zu finden.
Um das Reflexionsgesetz zu beweisen, wurden insgesamt 7 Messwerte für Einfalls- und Ausfallswinkel genommen. Diese sind in folgender Tabelle
aufgelistet.
\begin{table}
  \centering
  \caption{Einfalls- und Ausfallswinkel.}
  \label{tab:Aufgabe1}
  \begin{tabular}{c c c}
    \toprule
    Einfallswinkel $\alpha_1$ / $^{\circ}$& Ausfallswinkel $\alpha_2$ / $^{\circ}$ & Differenz $x$ / $^{\circ}$\\
    \midrule
    80 & 80.5 & 0.5 \\
    70 & 70.5 & 0.5\\
    60  &60.5 & 0.5\\
    50  &50 & 0\\
    40 & 40 & 0\\
    30 & 30 & 0\\
    20 & 20 & 0\\
    \bottomrule
  \end{tabular}
\end{table}
Die Winkel konnten mit einer Genauigkeit von bis zu $0,5^{\circ}$ gemessen werden. \\
Mit dieser Messungenauigkeit ergibt sich eine durchschnittliche Differenz des Einfalls- und
Ausfallswinkels von $x = 0,2143^{\circ} \pm 0,7071^{\circ}$. Die Meßunsicherheit wurde mithilfe der Gaußschen Fehlerfortpflanzung berechnet.

\subsection{Brechungsgesetz}
Um das Brechungsgesetz zu beweisen, wird der in der Durchführung beschriebene Versuchsaufbau verwendet. Da der Lichtstrahl von einem optisch dünneren
Medium (Luft, $n_1=1$) in ein dickeres übergeht, gilt das Snelliussche Brechungsgesetz \autoref{eqn:Snellius} mit $n_1=1$ und $n_2=n$.\\
Um die Gesetzmäßigkeiten zu überprüfen, wurden jeweils 7 Einfalls- und Brechungswinkel gemessen, die sich in nachfolgender Tabelle finden lassen.

\begin{table}
  \centering
  \caption{Einfalls- und Brechungswinkel.}
  \label{tab:Aufgabe2}
  \begin{tabular}{c c c}
    \toprule
    Einfallswinkel $\alpha$ / $^{\circ}$& Brechungswinkel $\beta$ / $^{\circ}$ & Brechungsindex $n$\\
    \midrule
    70 & 39 & 1,493\\
    60 & 35,5& 1,491\\
    50 & 30,5& 1,510\\
    40 & 25& 1,521\\
    30 & 19& 1,536\\
    20 & 13& 1,520\\
    10 & 6.5& 1,534\\
    \bottomrule
  \end{tabular}
\end{table}

\subsubsection{Brechungsindex Plexiglas}

Die Messwerte ergeben mit \autoref{eqn:Snellius} und der Gaußschen Fehlerfortpflanzung den Brechungsindex $n = 1,515 \pm 0,035$.\\
Für die Gaußsche Fehlerfortpflanzung wurde folgende Formel benutzt:
\begin{align}
  \label{eqn:Gauss}
  \Delta n = \sqrt{\left(\frac{\cos(\={\alpha})}{\sin(\={\beta})}\right)^2 \cdot \Delta {\alpha}^2 + \left( - \frac{\sin(\={\alpha})\cos(\={\beta})}{\sin(\={\beta})^2}\right)^2 \cdot \Delta {\beta}^2}
\end{align}

\subsubsection{Lichtgeschwindigkeit $v$ im Plexiglas}

Die Ausbreitungsgeschwindigkeit von Licht verhält sich antiproportional zum Brechungsindex und lässt sich nach
\begin{align}
  \label{eqn:Geschw}
  v = \frac{c}{n}
\end{align}
bestimmen. Wobei $v$ die neue Ausbreitungsgeschwindigkeit, $c$ die Lichtgeschwindigkeit im Vakuum und $n$ der Brechungsindex ist.\\
Die neue Ausbreitungsgeschwindigkeit ist demnach $197.882.810,6 \pm 4.571.559,1 \, \frac{\textrm{m}}{\textrm{s}}$.\\

\subsection{Planparallele Platten}

Für diesen Versuch wurden dieselben Messwerte genommen wie für den Brechungsindex. Die Messwerte lassen sich in \autoref{tab:Aufgabe2} wiederfinden.
Es werden jeweils nur die ersten 5 Messwerte verwendet.\\

\subsubsection{Berechnung mit Messwerten}
Zunächst wird der Strahlversatz mithilfe der Messwerte und der Formel
\begin{align}
  \label{eqn:Geschw}
  s = d \frac{\sin(\alpha - \beta)}{\cos(\beta)}
\end{align}
berechnet.
Dazu wird die Gaußsche Fehlerfortpflanzung
\begin{align}
  \label{eqn:Gauss2}
  \Delta s = \sqrt{\left(\frac{\cos(\alpha - \beta)}{\cos(\beta)}\right)^2 \Delta \alpha^2 + \left(-\frac{\sin(\beta)\sin(\beta - \alpha) + \cos(\beta)\cos(\beta - \alpha)}{\cos(\beta)^2}\right)^2 \Delta \beta^2}
\end{align}
verwendet.

\begin{table}
  \centering
  \caption{Einfalls-, Brechungswinkel und Strahlversatz mit Messwerten.}
  \label{tab:Aufgabe3}
  \begin{tabular}{c c c}
    \toprule
    Einfallswinkel $\alpha$ / $^{\circ}$& Brechungswinkel $\beta$ / $^{\circ}$ & Strahlversatz $s$ / m\\
    \midrule
    70 & 39 & 0,0388 $\pm 0,0006$ \\
    60 & 35,5& 0,0298 $\pm 0,0007$\\
    50 & 30,5& 0,0227 $\pm 0,0007$\\
    40 & 25& 0,0167 $\pm 0,0007$\\
    30 & 19& 0,0118 $\pm 0,0007$\\
    \bottomrule
  \end{tabular}
\end{table}

Der berechnete Strahlversatz lässt sich in der Tabelle finden.\\

\subsubsection{Berechnung mit Brechungsindex}

Nun wird der Brechungswinkel mithilfe des Brechungsindex aus dem vorherigen Versuch berechnet und dann
der Strahlversatz berechnet. Es gilt $n = 1,515 \pm 0,035$.
\begin{table}
  \centering
  \caption{Einfalls-, Brechungswinkel und Strahlversatz mit Brechungsindex.}
  \label{tab:Aufgabe3}
  \begin{tabular}{c c c}
    \toprule
    Einfallswinkel $\alpha$ / $^{\circ}$& Brechungswinkel $\beta$ / $^{\circ}$ & Strahlversatz $s$ / m\\
    \midrule
    70 & 38,34 $\pm 0,018$ & 0,039 $\pm 0,0006$\\
    60 & 34,86 $\pm 0,016$ & 0,030 $\pm 0,0007$\\
    50 & 30,37 $\pm 0,014$ & 0,023 $\pm 0,0007$\\
    40 & 25,11 $\pm 0,011$ & 0,017 $\pm 0,0006$\\
    30 & 19,27 $\pm 0,008$ & 0,012 $\pm 0,0005$\\
    \bottomrule
  \end{tabular}
\end{table}

\subsection{Prisma}