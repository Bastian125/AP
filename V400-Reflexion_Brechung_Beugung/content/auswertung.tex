\section{Auswertung}
\label{sec:Auswertung}

\subsection{Vorbereitungsaufgaben}
Im Rahmen der Vorbereitungsaufgaben wird im Folgenden der Brechungsindex verschiedener Materialien recherchiert und die Gitterkonstanten verschiedener Gitter ausgerechnet.
\subsubsection{Brechungsindex verschiedener Materialien}
\begin{table}
  \centering
  \caption{Brechungsindex verschiedener Materialien.}
  \label{tab:Brechungsindex}
  \begin{tabular}{c c}
    \toprule
    Material & Brechungsindex $n$ \\
    \midrule
    Luft      & 1,0003 \cite{Wien} \\
    Wasser    & 1,333 \cite{Wien} \\
    Kronglas  & 1,510 \cite{Spektrum} \\
    Plexiglas & 1,492 \cite{Wiki} \\
    Diamant   & 2,417 \cite{Wien} \\
    \bottomrule
  \end{tabular}
\end{table}
\subsubsection{Gitterkonstanten $d$}
\begin{itemize}
  \item 600 Linien/mm; $d=\frac{10}{6} $ \textmu m
  \item 300 Linien/mm; $d=\frac{10}{3} $ \textmu m
  \item 100 Linien/mm; $d=\frac{10}{1} $ \textmu m
\end{itemize}

\subsection{Reflexionsgesetz}
Bei diesem Versuch wurde der grüne Laser, Vorlage A und ein Spiegel verwendet. Der genauere Aufbau ist bei dem Kapitel Durchführung zu finden.
Um das Reflexionsgesetz zu beweisen, wurden insgesamt 7 Messwerte für Einfalls- und Ausfallswinkel genommen. Diese sind in folgender Tabelle
aufgelistet.
\begin{table}
  \centering
  \caption{Einfalls- und Ausfallswinkel.}
  \label{tab:Aufgabe1}
  \begin{tabular}{c c c}
    \toprule
    Einfallswinkel $\alpha_1$ / $^{\circ}$& Ausfallswinkel $\alpha_2$ / $^{\circ}$ & Differenz $x$ / $^{\circ}$\\
    \midrule
    80 $\pm 0,5$ & 80,5 $\pm 0,5$ & 0,5 \\
    70 $\pm 0,5$& 70,5 $\pm 0,5$& 0,5\\
    60  $\pm 0,5$&60,5 $\pm 0,5$& 0,5\\
    50  $\pm 0,5$&50 $\pm 0,5$& 0\\
    40 $\pm 0,5$& 40 $\pm 0,5$& 0\\
    30 $\pm 0,5$& 30 $\pm 0,5$& 0\\
    20 $\pm 0,5$& 20 $\pm 0,5$& 0\\
    \bottomrule
  \end{tabular}
\end{table}
Die Winkel konnten mit einer Genauigkeit von bis zu $0,5^{\circ}$ gemessen werden. \\
Mit dieser Messungenauigkeit ergibt sich eine durchschnittliche Differenz des Einfalls- und
Ausfallswinkels von $x = 0,2143^{\circ} \pm 0,7071^{\circ}$. Die Meßunsicherheit wurde mithilfe Gaußschen Fehlerfortpflanzung berechnet.

\subsection{Brechungsgesetz}
Um das Brechungsgesetz zu beweisen, wird der in der Durchführung beschriebene Versuchsaufbau verwendet. Da der Lichtstrahl von einem optisch dünneren
Medium (Luft, $n_1=1$) in ein dickeres übergeht, gilt das Snelliussche Brechungsgesetz \autoref{eqn:Snellius} mit $n_1=1$ und $n_2=n$.\\
Um die Gesetzmäßigkeiten zu überprüfen, wurden jeweils 7 Einfalls- und Brechungswinkel gemessen, die sich in nachfolgender Tabelle finden lassen.

\begin{table}
  \centering
  \caption{Einfalls- und Brechungswinkel.}
  \label{tab:Aufgabe2}
  \begin{tabular}{c c c}
    \toprule
    Einfallswinkel $\alpha$ / $^{\circ}$& Brechungswinkel $\beta$ / $^{\circ}$ & Brechungsindex $n$\\
    \midrule
    70 $\pm 0,5$ & 39 $\pm 0,5$ & 1,493 $\pm 0,017$\\
    60 $\pm 0,5$ & 35,5 $\pm 0,5$& 1,491 $\pm 0,0197$\\
    50 $\pm 0,5$ & 30,5 $\pm 0,5$& 1,510 $\pm 0,025$\\
    40 $\pm 0,5$ & 25 $\pm 0,5$& 1,521 $\pm 0,032$\\
    30 $\pm 0,5$ & 19 $\pm 0,5$& 1,536 $\pm 0,028$\\
    20 $\pm 0,5$ & 13 $\pm 0,5$& 1,520 $\pm 0,068$\\
    10 $\pm 0,5$ & 6,5 $\pm 0,5$& 1,534 $\pm 0,139$\\
    \bottomrule
  \end{tabular}
\end{table}

\subsubsection{Brechungsindex Plexiglas}

Die Messwerte ergeben mit \autoref{eqn:Snellius} und der Gaußschen Fehlerfortpflanzung den Brechungsindex $n = 1,515 \pm 0,035$.\\
Für die Gaußsche Fehlerfortpflanzung wurde folgende Formel benutzt:
\begin{align}
  \label{eqn:Gauss}
  \Delta n = \sqrt{\left(\frac{\cos(\={\alpha})}{\sin(\={\beta})}\right)^2 \cdot \Delta {\alpha}^2 + \left( - \frac{\sin(\={\alpha})\cos(\={\beta})}{\sin(\={\beta})^2}\right)^2 \cdot \Delta {\beta}^2}
\end{align}

\subsubsection{Lichtgeschwindigkeit $v$ im Plexiglas}

Die Ausbreitungsgeschwindigkeit von Licht verhält sich antiproportional zum Brechungsindex und lässt sich nach
\begin{align}
  \label{eqn:Geschw}
  v = \frac{c}{n}
\end{align}
bestimmen. Wobei $v$ die neue Ausbreitungsgeschwindigkeit, $c$ die Lichtgeschwindigkeit im Vakuum und $n$ der Brechungsindex ist.\\
Die neue Ausbreitungsgeschwindigkeit ist demnach $197.882.810,6 \pm 4.571.559,1 \, \frac{\textrm{m}}{\textrm{s}}$.\\

\subsection{Planparallele Platten}

Für diesen Versuch wurden dieselben Messwerte genommen wie für den Brechungsindex. Die Messwerte lassen sich in \autoref{tab:Aufgabe2} wiederfinden.
Es werden jeweils nur die ersten 5 Messwerte verwendet.\\

\subsubsection{Berechnung mit Messwerten}
Zunächst wird der Strahlversatz mithilfe der Messwerte und der Formel
\begin{align}
  \label{eqn:Geschw}
  s = d \frac{\sin(\alpha - \beta)}{\cos(\beta)}
\end{align}
berechnet.
Dazu wird die Gaußsche Fehlerfortpflanzung
\begin{align}
  \label{eqn:Gauss2}
  \Delta s = \sqrt{\left(\frac{\cos(\alpha - \beta)}{\cos(\beta)}\right)^2 \Delta \alpha^2 + \left(-\frac{\sin(\beta)\sin(\beta - \alpha) + \cos(\beta)\cos(\beta - \alpha)}{\cos(\beta)^2}\right)^2 \Delta \beta^2}
\end{align}
verwendet.

\begin{table}
  \centering
  \caption{Einfalls-, Brechungswinkel und Strahlversatz mit Messwerten.}
  \label{tab:Aufgabe4a}
  \begin{tabular}{c c c}
    \toprule
    Einfallswinkel $\alpha$ / $^{\circ}$& Brechungswinkel $\beta$ / $^{\circ}$ & Strahlversatz $s$ / m\\
    \midrule
    70 $\pm 0,5$ & 39 $\pm 0,5$ & 0,0388 $\pm 0,0006$ \\
    60 $\pm 0,5$ & 35,5 $\pm 0,5$ & 0,0298 $\pm 0,0007$\\
    50 $\pm 0,5$ & 30,5 $\pm 0,5$ & 0,0227 $\pm 0,0007$\\
    40 $\pm 0,5$ & 25 $\pm 0,5$ & 0,0167 $\pm 0,0007$\\
    30 $\pm 0,5$ & 19 $\pm 0,5$ & 0,0118 $\pm 0,0007$\\
    \bottomrule
  \end{tabular}
\end{table}

Der berechnete Strahlversatz lässt sich in der \autoref{tab:Aufgabe4a} finden.\\

\subsubsection{Berechnung mit Brechungsindex}

Nun wird der Brechungswinkel mithilfe des Brechungsindex aus dem vorherigen Versuch berechnet und dann
der Strahlversatz berechnet. Es gilt $n = 1,515 \pm 0,035$.
\begin{table}
  \centering
  \caption{Einfalls-, Brechungswinkel und Strahlversatz mit Brechungsindex.}
  \label{tab:Aufgabe4b}
  \begin{tabular}{c c c}
    \toprule
    Einfallswinkel $\alpha$ / $^{\circ}$& Brechungswinkel $\beta$ / $^{\circ}$ & Strahlversatz $s$ / m\\
    \midrule
    70 & 38,335 $\pm 0,018$ & 0,0391 $\pm 0,0006$\\
    60 & 34,864 $\pm 0,016$ & 0,0303 $\pm 0,0007$\\
    50 & 30,374 $\pm 0,014$ & 0,0228 $\pm 0,0007$\\
    40 & 25,105 $\pm 0,011$ & 0,0166 $\pm 0,0006$\\
    30 & 19,271 $\pm 0,008$ & 0,0115 $\pm 0,0005$\\
    \bottomrule
  \end{tabular}
\end{table}

Der berechnete Strahlversatz lässt sich in der \autoref{tab:Aufgabe4b} finden.\\

\subsection{Prisma}

In diesem Versuch wird die Ablenkung $\delta$ untersucht, die ein Lichtstrahl beim Durchgang durch ein Prisma erfährt.
Der Versuch wird wie beschrieben durchgeführt und jeweils fünf Einfallswinkel $\alpha_1$ und Ausfallswinkel $\alpha_2$ für einen 
grünen und roten Laser gemessen.\\
Die Ablenkung berechnet sich nach $\delta = (\alpha_1 + \alpha_2) - (\beta_1 + \beta_2)$.\\
Zunächst wird der Brechungswinkel im Prisma mithilfe von \autoref{eqn:Snellius} und $n=1,510$ bestimmt.

\subsubsection{Grüner Laser}

Die Messwerte und berechneten Werte für den grünen Laser lassen sich in  \autoref{tab:Aufgabe5a1} und die Ablenkung in \autoref{tab:Aufgabe5a2} finden.

\begin{table}
  \centering
  \caption{Einfalls-, Aufallswinkel beim grünen Laser.}
  \label{tab:Aufgabe5a1}
  \begin{tabular}{c c c c}
    \toprule
    Einfallswinkel 2 $\alpha_1$ / $^{\circ}$ & Brechungswinkel $\beta_1$ / $^{\circ}$ & Einfallswinkel 2 $\beta_2$ / $^{\circ}$ & Ausfallswinkel $\alpha_2$ / $^{\circ}$ \\
    \midrule
    40 $\pm 0,5$ & 25,194 $\pm 0,005$ & 34,806 $\pm 0,005$ & 60,5 $\pm 0,5$ \\
    45 $\pm 0,5$ & 27,923 $\pm 0,005$ & 32,077 $\pm 0,005$ & 55 $\pm 0,5$ \\
    50 $\pm 0,5$ & 30,485 $\pm 0,004$ & 29,515 $\pm 0,004$ & 50 $\pm 0,5$ \\
    55 $\pm 0,5$ & 32,853 $\pm 0,004$ & 27,147 $\pm 0,004$ & 45 $\pm 0,5$ \\
    60 $\pm 0,5$ & 34,997 $\pm 0,004$ & 25,003 $\pm 0,004$ & 40 $\pm 0,5$ \\
    \bottomrule
  \end{tabular}
\end{table}

Die Gaußsche Fehlerfortpflanzung des Brechungswinkel berechnet sich zu:
\begin{align}
  \label{eqn:Gauss3}
  \Delta \beta_1 = \left(\frac{\cos(\alpha_1)}{n_{\textrm{Kron}}\cdot \sqrt{1-\frac{\sin(\alpha_1)^2}{n_{\textrm{Kron}}^2}}} \right) \Delta \alpha_1
\end{align}
Der zweite Einfallswinkel $\beta_2$ lässt sich aus der Winkelbeziehung des Prismas ableiten: $\beta_2 = 60^{\circ} - \beta_1$

\begin{table}
  \centering
  \caption{Ablenkung beim grünen Laser.}
  \label{tab:Aufgabe5a2}
  \begin{tabular}{c}
    \toprule
    Ablenkung $\delta$ / $^{\circ}$ \\
    \midrule
    40,5 $\pm 0,707$ \\
    40 $\pm 0,707$\\
    40 $\pm 0,707$\\
    40 $\pm 0,707$\\
    40 $\pm 0,707$\\
    \bottomrule
  \end{tabular}
\end{table}

Hiermit ergibt sich nun eine Ablenkung $\delta$ von $\delta = 40,1 \pm 0,4 ^{\circ}$.

\subsubsection{Roter Laser}

Die Rechnung zum roten Laser erfolgt analog zum grünen Laser. Die Messwerte und berechneten werden finden sich in \autoref{tab:Aufgabe5b1} und \autoref{tab:Aufgabe5b2}.

\begin{table}
  \centering
  \caption{Einfalls-, Aufallswinkel beim roten Laser.}
  \label{tab:Aufgabe5b1}
  \begin{tabular}{c c c c}
    \toprule
    Einfallswinkel 2 $\alpha_1$ / $^{\circ}$ & Brechungswinkel $\beta_1$ / $^{\circ}$ & Einfallswinkel 2 $\beta_2$ / $^{\circ}$ & Ausfallswinkel $\alpha_2$ / $^{\circ}$ \\
    \midrule
    40 $\pm 0,5$ & 25,194 $\pm 0,005$ & 34,806 $\pm 0,005$ & 61 $\pm 0,5$ \\
    45 $\pm 0,5$ & 27,923 $\pm 0,005$ & 32,077 $\pm 0,005$ & 54 $\pm 0,5$ \\
    50 $\pm 0,5$ & 30,485 $\pm 0,004$ & 29,515 $\pm 0,004$ & 49 $\pm 0,5$ \\
    55 $\pm 0,5$ & 32,853 $\pm 0,004$ & 27,147 $\pm 0,004$ & 44 $\pm 0,5$ \\
    60 $\pm 0,5$ & 34,997 $\pm 0,004$ & 25,003 $\pm 0,004$ & 40 $\pm 0,5$ \\
    \bottomrule
  \end{tabular}
\end{table}

\begin{table}
  \centering
  \caption{Ablenkung beim roten Laser.}
  \label{tab:Aufgabe5b2}
  \begin{tabular}{c}
    \toprule
    Ablenkung $\delta$ / $^{\circ}$ \\
    \midrule
    41 $\pm 0,707$ \\
    39 $\pm 0,707$\\
    39 $\pm 0,707$\\
    39 $\pm 0,707$\\
    40 $\pm 0,707$\\
    \bottomrule
  \end{tabular}
\end{table}

Hiermit ergibt sich nun eine Ablenkung $\delta$ von $\delta = 39,6 \pm 0,4 ^{\circ}$.

\subsection{Beugung am Gitter}

In diesem Versuch soll die Wellenlänge der zwei Laser anhand der Beugung an einem Gitter ermittelt werden.
Dazu wird der beschriebene Versuchsaufbau verwendet und die Ablenkwinkel $\phi$ der jeweiligen Beugungsordnungen $k$
gemessen. Das sind bei den verschiedenen Gittern unterschiedlich viele Messwerte.\\
Die Wellenlänge wird anschließend mithilfe von \autoref{eqn:maxima} berechnet.

\subsubsection{600 Linien / mm}

Bei diesem Versuchsteil wird ein Gitter mit 600 Linien / mm verwendet. Die Gitterkonstante zu diesem Gitter lässt sich in den Vorbereitungsaufgaben finden
und die dazugehörigen Messwerte sind in \autoref{tab:Aufgabe6a}.

\begin{table}
  \centering
  \caption{Beugungsmaxima bei 600 Linien / mm.}
  \label{tab:Aufgabe6a}
  \begin{tabular}{c c c c c}
    \toprule
    Beugungsordnung $k$ & $\textrm{Ablenkwinkel}_{\textrm{rot}}$ / $^{\circ}$ & $\lambda_{\textrm{rot}}$ / nm & $\textrm{Ablenkwinkel}_{\textrm{grün}}$ / $^{\circ}$ & $\lambda_{\textrm{rot}}$ / nm\\
    \midrule
    1 & 15 $\pm 0,5$ & 431,37 $\pm 14,01$ & 12 $\pm 0,5$ & 346,52 $\pm 14,18$\\
    \bottomrule
  \end{tabular}
\end{table}

\subsubsection{300 Linien / mm}

Nun wird ein Gitter mit 300 Linien / mm verwendet. Die Gitterkonstante zu diesem Gitter lässt sich ebenfalls in den Vorbereitungsaufgaben finden
und die dazugehörigen Messwerte sind in \autoref{tab:Aufgabe6b}.

\begin{table}
  \centering
  \caption{Beugungsmaxima bei 300 Linien / mm.}
  \label{tab:Aufgabe6b}
  \begin{tabular}{c c c c c}
    \toprule
    Beugungsordnung $k$ & $\textrm{Ablenkwinkel}_{\textrm{rot}}$ / $^{\circ}$ & $\lambda_{\textrm{rot}}$ / nm & $\textrm{Ablenkwinkel}_{\textrm{grün}}$ / $^{\circ}$ & $\lambda_{\textrm{rot}}$ / nm\\
    \midrule
    1 & 12 $\pm 0,5$ & 693,04 $\pm 28,37$ & 6 $\pm 0,5$ & 348,43 $\pm 28,84$\\
    2 & 15 $\pm 0,5$ & 431,37 $\pm 14,01$ & 12 $\pm 0,5$ & 346,52 $\pm 14,18$\\
    3 & 19 $\pm 0,5$ & 361,74 $\pm 91,40$ & 19 $\pm 0,5$ & 361,74 $\pm 91,40$\\
    \bottomrule
  \end{tabular}
\end{table}

\subsubsection{100 Linien / mm}

Hier wird zuletzt Gitter mit 100 Linien / mm verwendet. Die Gitterkonstante zu diesem Gitter lässt sich in den Vorbereitungsaufgaben finden
und die dazugehörigen Messwerte sind in \autoref{tab:Aufgabe6c}.

\begin{table}
  \centering
  \caption{Beugungsmaxima bei 100 Linien / mm.}
  \label{tab:Aufgabe6c}
  \begin{tabular}{c c c c c}
    \toprule
    Beugungsordnung $k$ & $\textrm{Ablenkwinkel}_{\textrm{rot}}$ / $^{\circ}$ & $\lambda_{\textrm{rot}}$ / nm & $\textrm{Ablenkwinkel}_{\textrm{grün}}$ / $^{\circ}$ & $\lambda_{\textrm{rot}}$ / nm\\
    \midrule
    1 & 2,5 $\pm 0,5$ & 436,19 $\pm 86,92$ & 2 $\pm 0,5$ & 348,995 $\pm 86,95$\\
    2 & 5 $\pm 0,5$ & 435,78 $\pm 43,33$ & 4 $\pm 0,5$ & 348,78 $\pm 43,39$\\
    3 & 7,5 $\pm 0,5$ & 435,09 $\pm 28,75$ & 6 $\pm 0,5$ & 348,43 $\pm 28,84$\\
    4 & 10 $\pm 0,5$ & 434,12 $\pm 21,41$ & 8 $\pm 0,5$ & 347,93 $\pm 21,54$\\
    5 & 13 $\pm 0,5$ & 449,90 $\pm 16,95$ & 10 $\pm 0,5$ & 347,296 $\pm 17,14$\\
    \bottomrule
  \end{tabular}
\end{table}

Insgesamt ergeben all diese Mittelwerte die Wellenlängen $\lambda_{\textrm{rot}} = 454,99 \pm 13,65 \, \textrm{nm}$ und $\lambda_{\textrm{grün}} = 349,01 \pm 13,68 \, \textrm{nm}$.