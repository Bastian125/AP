\section{Durchführung}
\label{sec:Durchführung}
Es wird eine transparente Glasplatte verwendet, auf der sich im Halbkreis verschiebbare Laserdioden befinden.
Die untere Laserdiode emittiert grünes Licht mit der Wellenlänge $\lambda=532\,\unit{\nano\meter}$ und die
obere rotes Licht mit der Wellenlänge $\lambda=635\,\unit{\nano\meter}$. Auf der Mitte der Platte können
unterschiedliche optische Elemente montiert werden. Auf der anderen Seite der Platte ist ein Reflexionsschirm
zum Schutz vor dem Laserlicht befestigt. In disem Versuch ist immer Luft das optische dünnerer Medium mit der
Lichtgeschwindigkeit $c$ und dem Brechungsindex $n\approx1$.

\subsection{Reflexionsgesetz}
\label{sec:Reflexionsgesetz}
Ein Spiegel wird in der Mitte der Platte befestigt. Um die Einfalls- und Ausfallswinkel ablesen zu können,
wird eine Vorlage unter die Platte geschoben. Dann wird für sieben Winkel mit dem grünen Laser die
beiden Winkel gemessen.

\subsection{Brechungsgesetz}
\label{sec:Brechungsgesetz}
Nun wird eine planparallele Platte anstelle des Spiegels eingesetzt. Die Vorlage und der Laser bleiben
unverändert. Durch die Vorlage lassen sich sowohl der Einfallswinkel, als auch der Brechungswinkel
ablesen. Es wird für sieben verschiedene Winkel gemessen und der Strahlenversatz berechnet.

\subsection{Prisma}
\label{sec:Prisma}
Das Prisma wird nun an Stelle der planparallelen Platte angebracht und die Winkelvorlage gewechselt.
Weiterhin wird ein Transmissionsschirm, der mit einer Winkelskala versehen ist, am Ende der Vorlage befestigt.
Dann werden erst mit dem grünen Laser und später mit dem roten Laser die Ausfallswinkel $\alpha_{\symup{2}}$ für
fünf verschiedene Einfallswinkel $\alpha_{\symup{1}}$ im Bereich $10^{\circ} \leq \alpha_{\symup{1}} \leq 60^{\circ}$
gemessen.

\subsection{Gitter}
\label{sec:Gitter}
Die Vorlage bleibt unverändert, der Transmissionsschirm wird gewechselt und das Prisma wird durch ein Gitter ausgetauscht.
Dabei wird das Gitter am Ende der Platte platziert, so dass auch der grüne Laser trotz mangelnder Höhe auf das Gitter trifft.
Es wird mit beiden Lasern die $k$-ten Intensitätsmaxima samt Winkel des Beugungsmusters gleichzeitig gemessen.