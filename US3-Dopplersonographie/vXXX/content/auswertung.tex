\section{Auswertung}
\label{sec:Auswertung}

\subsection{Strömungsgeschwindigkeit in Abhängigkeit des Dopplerwinkels}

Bei diesem Versuchsteil werden bei unterschiedlichen Pumpengeschwindigkeiten und Einstrahlwinkeln die
Frequenzverschiebungen des ausgehenden und eingehenden Schalls untersucht um daraus die Strömungsgeschwindigkeit
zu berechnen.\\
Die Strömungsgeschwindigkeit in Abhängigkeit des Dopplerwinkels lässt sich durch \autoref{eqn:fdiff}
berechnen. Da nicht die Frequenzverschiebung, sondern die Geschwindigkeit gesucht ist, wird die Formel
nach $v$ umgestellt. In die resultierende Formel
\begin{equation*}
  v_{\mathrm{Str}} = \frac{\symup{\Delta}\nu\cdot c}{\nu_0\,\symup{cos}\,\alpha} 
\end{equation*}
werden dann alle aufgenommenen Messwerte eingesetzt und die Strömungsgeschwindigkeiten ausgerechnet.\\
Die Messwerte lassen sich in \autoref{tab:Teil1} finden. Die Frequenzverschiebungen wurden jeweils
bei Einstrahlwinkeln von $\theta= 15 ^{\circ}$, $\theta= 30 ^{\circ}$ und $\theta= 45 ^{\circ}$ untersucht.
Aus diesen Einstrahlwinkeln lässt sich mithilfe von \autoref{eqn:alpha} der Dopplerwinkel berechnen.\\
\begin{table}
  \centering
  \begin{tabular}{c | c | c | c}
    \toprule
    Pumpengeschwindigkeit / rpm & \multicolumn{3}{c}{Fr.-versch. $f_{\mathrm{diff}}$} \\
    \hline
     & $15^\circ$ & $30^\circ$ & $45^\circ$ \\
    \midrule
    3970     &       56     &     104     &    161 \\
    5000     &       90     &     161     &    312 \\
    6000     &       124    &     256     &    382\\
    7000     &       181    &     283     &    449\\
    9300     &       326    &     503    &    1167\\
    \bottomrule
  \end{tabular}
  \caption{Aufgenommene Messwerte zur Untersuchung der Strömungsgeschwindigkeit in Abhängigkeit des Dopplerwinkels.}
  \label{tab:Teil1}
\end{table}

Mithilfe der obigen Formel werden dann die Strömungsgeschwindigkeiten ausgerechnet und ein Plot erstellt,
wo $\frac{\Delta f}{\cos{\alpha}}$ gegen die Strömungsgeschwindigkeit aufgetragen wird. Das Ergebnis ist in 
\autoref{plt:Teil1} zu sehen.

\begin{figure}
  \centering
  \includegraphics{build/plot1_2.pdf}
  \caption{Die Frequenzverschiebungen geteilt durch den Cosinus des Dopplerwinkels aufgetragen gegen 
  die Strömungsgeschwindigkeit.}
  \label{plt:Teil1}
\end{figure}


\subsection{Strömungsprofil der Dopplerflüssigkeit}

Für diesen Versuchsteil werden für unterschiedliche Strömungsgeschwindigkeiten und Messtiefen jeweils die
Frequenzverschiebungen, die Streuintensitäten des Ultraschalls und die Momentangeschwindigkeit der Flüssigkeit
bei einem Einstrahlwinkel von $\theta = 15^{\circ}$ aufgenommen.\\
Als Strömungsgeschwindigkeiten werden 4200 rpm (ca. 45\% der Gesamtleistung) und 6400 rpm (ca. 70\% der
Gesamtleistung) gewählt. Die aufgezeichneten Messwerte sind jeweils in \autoref{tab:Teil2_45} und 
\autoref{tab:Teil2_70} zu finden.\\

\begin{table}
  \centering
  \begin{tabular}{c | c | c | c}
    \toprule
    Messtiefe $l$ / $\si{\micro\second}$ & Fr.-versch. $f_{\mathrm{diff}}$ / $\si{\hertz}$ & Strömungsgeschw. $v$ / $\si{\centi\meter\per\second}$ & Intensität / $\si{\kilo\volt\squared\per\second}$ \\
    \midrule
    13.5        &    0      &     0         &      29\\
    14          &    81     &     23.9      &      33\\
    14.5        &    71     &     23.9      &      35\\
    15          &    74     &     23.9      &      34\\
    15.5        &    70     &     21.2      &      43\\
    16          &    68     &     15.9      &      33\\
    16.5        &    0      &     0         &      18\\
    17          &    0      &     0         &      13\\
    \bottomrule
  \end{tabular}
  \caption{Aufgenommenen Messwerte bei einer Durchlaufgeschwindigkeit des Wassers von 4200 rounds per minute.}
  \label{tab:Teil2_45}
\end{table}

\begin{table}
  \centering
  \begin{tabular}{c | c | c | c}
    \toprule
    Messtiefe $l$ / $\si{\micro\second}$ & Fr.-versch. $f_{\mathrm{diff}}$ / $\si{\hertz}$ & Strömungsgeschw. $v$ / $\si{\centi\meter\per\second}$ & Intensität / $\si{\kilo\volt\squared\per\second}$ \\
    \midrule
    12          &    0      &     0          &     14\\
    12.5        &    0      &     0          &     24\\
    13          &    155    &     38.5       &     32\\
    13.5        &    188    &     45.1       &     34\\
    14          &    183    &     53.0       &     34\\
    14.5        &    181    &     55.7       &     42\\
    15          &    180    &     55.7       &     49\\
    15.5        &    173    &     45.1       &     55\\
    16          &    148    &     35.8       &     64\\
    16.5        &    145    &     26.5       &     58\\          
    17          &    0      &     0          &     25\\
    17.5        &    0     &      0          &     11\\
    18          &    0     &      0          &     8\\
    18.5        &    0     &      0          &     8\\
    19          &    0     &      0          &     6\\
    \bottomrule
  \end{tabular}
  \caption{Aufgenommenen Messwerte bei einer Durchlaufgeschwindigkeit des Wassers von 6400 rounds per minute.}
  \label{tab:Teil2_70}
\end{table}

Für die zwei Strömungsgeschwindigkeit werden jeweils Diagramme erstellt, wo die Streuintensität
und die Momentangeschwindigkeit gegen die Messtiefe aufgetragen werden.\\

\begin{figure}
  \centering
  \includegraphics{build/plot2_1.pdf}
  \caption{Die Streuintensität und Momentangeschwindigkeit audgetragen gegen die Messtiefe bei
  einer Strömungsgeschwindigkeit von 4200 rpm.}
  \label{fig:stroemung}
\end{figure}

\begin{figure}
  \centering
  \includegraphics{build/plot2_2.pdf}
  \caption{Die Streuintensität und Momentangeschwindigkeit audgetragen gegen die Messtiefe bei
  einer Strömungsgeschwindigkeit von 6400 rpm.}
  \label{fig:stroemung}
\end{figure}