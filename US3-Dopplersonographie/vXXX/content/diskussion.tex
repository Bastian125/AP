\section{Diskussion}
\label{sec:Diskussion}

\subsection{Strömungsgeschwindigkeit in Abhängigkeit des Dopplerwinkels}

Es wird der Zusammenhang zwischen dem Dopplerwinkel und der Strömungsgeschwindigkeit $v_{\mathrm{Str}}$
untersucht.\\
Dafür wurde $\frac{\Delta \mu}{\cos{\alpha}}$ gegen die Strömungsgeschwindigkeit in einem Diagramm
aufgetragen. Da $\Delta \mu$ von $\cos{\alpha}$, $v_{\mathrm{Str}}$ und weiteren Konstanten abhängt,
muss sich ein linearer Zusammenhang ergeben, wo die Messwerte bei den drei Einfallswinkeln jeweils
gruppiert sind.\\
Der lineare Zusammenhang lässt sich wie erwartet anhand von \autoref{plt:Teil1} bestätigen. Allerdings
ist bei höheren Strömungsgeschwindigkeiten eine größere Streuung der Messwerte zu erkennen. Die Messwerte
zu den jeweiligen Winkeln sind also nicht mehr gruppiert.\\
Dies lässt sich dadurch erklären, dass beim Aufnehmen der Messwerte starke Schwankungen der aufgenommenen 
Frequenzen zu beobachten war, wodurch eine starke Ungenauigkeit resultiert.\\

\subsection{Strömungsprofil der Dopplerflüssigkeit}

Wie in den Grafiken abzulesen ist, wird zu bestimmten Messtiefen keine Streuintensität oder
Momentangeschwindigkeit aufgenommen. Das liegt daran, dass sich diese Messtiefen außerhalb der Röhre
befinden. Daher wird hier kein Strom und Intensität gemessen.\\
Ansonsten zeigen die dargestellten Strömungsprofile die typischen Merkmale einer laminaren Strömung 
auf. Die Strömungsgeschwindigkeiten weisen bei circa 15 µs einen Peak auf, was in etwa der Mitte der
Röhre entspricht und sinkt dann leicht zu den Rändern der Röhre hin.\\
Der im Vergleich zur Momentangeschwindigkeit verschobenen Peak der Streuintensität lässt sich durch
eine größere Frequenzverschiebung bei größeren Strömungsgeschwindigkeiten erklären.\\

\addsec{Anhang}
\label{Anhang}