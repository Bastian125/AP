\section{Durchführung}
\label{sec:Durchführung}

Es wird ein Kreislauf aus Strömungsrohren mit einem Gemisch aus Wasser, Glycerin und Glaskugeln aufgebaut
und an eine Pumpe angeschlossen. Weiterhin wird eine $\SI{2}{\mega\hertz}$-Ultraschallsonde an einen
Ultraschall-Doppler-Generator angeschlossen, der zur Auswertung an einen Computer mit dem Programm
FlowView verbunden ist. Um die Winkel zu variieren wird ein Dopplerprisma mit genug Ultraschallgel auf dem
Strömungsrohr angebracht. Zuerst werden die Frequenzverschiebungen für fünf unterschiedliche Pumpfrequenzen und
drei Winkelen von $15^{\circ}$, $30^{\circ}$ und $45^{\circ}$ gemessen. Dazu wird am Ultraschallgenerator das
Sample Volume auf Large gestellt. Danach für einen Winkel von $15^{\circ}$ die Strömungsgeschwindigkeit bei
$4200\,\symup{rpm}$ und $6400\,\symup{rpm}$, was ungefähr $\SI{45}{\%}$ bzw. $\SI{70}{\%}$ der maximalen
Pumpleistung entspricht, bestimmt. Dazu wird das Sample Volume auf Small gestellt und mit dem Depthregler
die Messtiefe von $\SI{12}{\micro\second}$ bis $\SI{19}{\micro\second}$ gesteigert. $\SI{4}{\micro\second}$
entsprechen dabei ungefähr $\SI{10}{\milli\meter}$ in Acryl und $\SI{4}{\micro\second}$ entrsprechen ungefähr
$\SI{6}{\milli\meter}$ in der Dopplerflüssigkeit.