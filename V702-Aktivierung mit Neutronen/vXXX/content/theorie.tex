\section{Zielsetzung}
In diesem Versuch soll die Halbwertszeit $T$ verschiedener instabiler Isotope und Isotopengemische 
untersucht werden.

\section{Theorie}
\label{sec:Theorie}

Atomkerne sind instabil, wenn das Verhältnis der Neutronen- und Protonenanzahl außerhalb eines atomspezifischen
Stabilitätsbereichs liegt. Dieser Akzeptanzrahmen liegt meistens zwischen 20 und 50\% mehr
Neutronen als Ordnungszahl Z.\\
Ein instabiler Atomkern kann mit von Atom zu Atom unterschiedlicher Wahrscheinlichkeit zu verschiedenen
Zerfallsprodukten zerfallen. Diese Zerfallsprodukte können Atome und/oder Teilchen sein.\\
Da nicht genau vorhergesagt werden kann, ob und wann ein instabiler Kern zerfällt, wird eine andere Größe verwendet, 
um die Wahrscheinlichkeit eines Zerfalls abschätzen zu können. Der Zeitraum, in dem die Hälfte der 
vorhandenen instabilen Kerne zerfallen ist, nennt sich Halbwertszeit $T$.\\

\subsection{Aktivierung mittels Neutronenbeschuss}

In diesem Versuch werden Atomkerne mit Neutronen beschossen, um diese in einen angeregten Zustand zu überführen
und anschließend deren Zerfälle zu beobachten. Diese angeregten Kerne heißen auch Zwischenkern oder Compoundkern.\\
Die Neutronen werden von den stabilen Atomkernen mit einer gewissen Wahrscheinlichkeit eingefangen, welche vom 
Wirkungsquerschnitt abhängt. Der Wirkungsquerschnitt $\sigma$ ist hierbei die Fläche, die ein Kern besitzen müsste, 
wenn jedes diese Fläche treffende Neutron eingefangen werden würde.\\
Der Wirkungsquerschnitt wird in $10^{-24} \, \si{\centi\metre\squared} \coloneqq 1 \, \mathrm{barn}$ angegeben und
lässt sich durch die Formel
\begin{equation}
    \label{eqn:Sigma}
    \sigma = \frac{u}{nKd}
\end{equation}
beschreiben.\\
Der Wirkungsquerschnitt für den Neutroneneinfang ist zudem stark abhängig von der Geschwindigkeit der Neutronen und
somit dessen kinetischen Energie. \\
Die Wellenlänge $\lambda$ des Neutrons ist antiproportional zu dessen Geschwindigkeit $v$ $ \left(\lambda \propto \frac{1}{v}\right)$.
Wenn die Geschwindigkeit hoch genug und die Wellenlänge somit klein gegen den Kernradius R $\left(\approx 10^{-12} \si{\centi\metre}\right)$ ist,
können geometrische Überlegungen auf die Wechselwirkung angewendet werden. Bei langsamen Neutronen $\left(R \leq \lambda\right)$ sind solche
Überlegungen durch Interferenzeffekte nicht möglich.\\
Breit und Wigner haben eine Formel aufgestellt, die den Zusammenhang zwischen Wirkungsquerschnitt und Neutronenenergie
darstellt:
\begin{equation}
    \label{eqn:SigmaE}
    \sigma (E) = \sigma_0 \sqrt{\frac{E_{r_i}}{E}} \frac{\tilde{c}}{(E - E_{r_i})^2 + \tilde{c}} .
\end{equation}
$E$ ist hier die Neutronenenergie, $E_{r_i}$ die Energieniveaus der Zwischenkerne und $\tilde{c}$ und $\sigma_0$ sind charakteristische Konstanten.\\
In der Formel lässt sich ablesen, dass der Wirkungsquerschnitt maximal ist, wenn die Neutronenenergie gleich der Höhe des Energieniveaus
des Zwischenkerns ist.\\
Wenn $E \ll E_r$, dann kann der quadrierte Term in der Formel als konstant angenommen werden und es ergibt sich die Proportionalität
\begin{equation*}
    \sigma \propto \frac{1}{v}.
\end{equation*}
Somit ist die Einfangswahrscheinlichkeit größer, je kleiner die Energie des Neutrons ist. Daher werden für den Versuch vorzugsweise
langsame Neutronen verwendet.\\

\subsection{Erzeugung niederenergetischer freier Neutronen}
Neutronen sind als freies Teilchen instabil und zerfallen nach etwa einer Viertelstunde. Daher kommen sie nicht in der Natur vor und müssen
während des Versuchs erzeugt werden.\\
Wenn $^9 \mathrm{Be}$-Kerne mit Heliumkernen ($\alpha$-Teilchen) beschossen werden, wird dieser Kern zu einem Kohlenstoffkern und es wird
ein Neutron emittiert:
\begin{equation*}
    \isotope[9][4]{Be} + \isotope[4][2]{} \alpha \to \isotope[12][6]{C} + \isotope[1][0]{n}.
\end{equation*}
Die Heliumkerne stammen von dem $\alpha$-Zerfall von $\isotope[226][]{Ra}$-Kernen.\\
Um nun niederenergetische Neutronen zu erhalten, werden diese dazu gezwungen durch dicke Materieschichten aus leichten Atomen hindurch zu diffundieren. 
Hierbei verlieren die Neutronen durch die elastischen Stöße mit den anderen Teilchen an Energie. Nach dem Gesetz des elastischen
Stoßes ist die Energieübertragung umso größer, je kleiner die Massendifferenz der Teilchen ist. Also bietet sich das masseärmste
Atom - Wasserstoff - an.\\
Die Neutronenenergie nähert sich somit der mittleren kinetischen Energie der Moleküle der Umgebung an. Alle Neutronen mit dieser
spezifischen Energie und Geschwindigkeit werden als thermische Neutronen bezeichnet.\\

\subsection{\texorpdfstring{$\beta^-$}{Beta}-Zerfall und \texorpdfstring{$\gamma$}{Gamma}-Zerfall}

Wenn ein stabiler Kern mit einem Neutron beschossen und dieses absorbiert wird, ist die Energie des Compoundkerns um die kinetische
Energie und Bindungsenergie des Neutrons höher als die des Ursprungskerns. Diese Energie verteilt sich dann auf alle Nukleonen, welche
dadurch angeregt werden.\\
Wenn die Energie des einfallenden Neutrons klein genug ist, bewirkt die Verteilung der Energie im Kern, dass der Compoundkern das eingefallene
Neutron oder ein anderes Nukleon nicht abstoßen kann. \\
Stattdessen geht der Kern durch emittieren eines $\gamma$-Quants ($\gamma$-Zerfall) nach kurzer Zeit $(10^{-16}\si{s})$ wieder in einen stabileren
Zustand über:
\begin{equation*}
    \isotope[m][z]{A} + \isotope[1][0]{n} \to \isotope[m+1][z]{A*} \to \isotope[m+1][z]{A} + \gamma.
\end{equation*}
Der Compoundkern ist allerdings durch die erhöhte Neutronenanzahl immernoch nicht stabil. Unter Emission eines Elektrons ($\beta ^-$-Zerfall)
geht der instabile Compoundkern in einen stabilen Kern über:
\begin{equation*}
    \isotope[m+1][z]{A} \to \isotope[m+1][z+1]{C} + \beta ^- + \mathrm{E_{kin}} + \overline{\nu_e}.
\end{equation*}
Die entstehende überschüssige Masse wird in kinetische Energie des Elektrons und Antineutrinos umgewandelt.

\subsection{Halbwertszeit}
Die Zahl $N(t)$ der zum Zeitpunkt $t$ noch nicht zerfallenen Kerne ist durch
\begin{equation}
    \label{eqn:Zerfallsgesetz}
    N(t) = N_0 e^{- \lambda t}
\end{equation}
gegeben. $N_0$ ist hier die zum Beginn vorhandenen instabilen Kerne und $\lambda$ die Zerfallskonstante, welche die Wahrscheinlichkeit
eines bestimmten Zerfalls beschreibt.\\
Wie bereits in der Einleitung beschrieben gibt die Halbwertszeit $T$ an, nach welcher Zeit die Hälfte aller instabiler Kerne
zerfallen ist. Mit diesem Wissen und der \autoref{eqn:Zerfallsgesetz}, lässt sich folgende Formel formulieren:
\begin{equation*}
    \frac{1}{2}N_0 = N_0 e^{- \lambda T}.
\end{equation*}
Durch Umformen ergibt sich dann schließlich
\begin{equation}
    \label{eqn:T}
    T = \frac{\ln{2}}{\lambda} .
\end{equation}
Durch Messen von $N(t)$ in Abhängigkeit der Zeit könnte $\lambda$ oder $T$ ermittelt werden. Allerdings ist es sehr schwierig $N(t)$ zuverlässig 
zu messen.\\
Daher wird die Anzahl der zerfallenen Kerne $N_{\Delta t}(t)$ in einem Zeitintervall $\Delta t$ gemessen. Die Größe $N_{\Delta t}(t)$ wird
durch die Formel
\begin{equation*}
    N_{\Delta t}(t) = N(t) - N(t + \Delta t)
\end{equation*}
beschrieben.
Dieser Ausdruck kann wieder umgeschrieben werden zu
\begin{equation}
    \label{eqn:lnDelta}
    \ln{N_{\Delta t}(t)} = \ln{N_0 (1-e^{- \lambda \Delta t})} - \lambda t
\end{equation}
Durch eine lineare Ausgleichsrechnung, kann mithilfe von \autoref{eqn:lnDelta} $\lambda$ bestimmt werden.\\
Hierbei ist die Wahl des Zeitintervalls wichtig. Bei kleinem $\Delta t$ wird der statistische Fehler für $N_{\Delta t}(t)$
sehr groß und bei zu großem $\Delta t$ der statistische Fehler für $\lambda$.