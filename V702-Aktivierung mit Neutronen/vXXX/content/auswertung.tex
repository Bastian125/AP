\section{Auswertung}
\label{sec:Auswertung}
\subsection{Nullrate}
\label{sec:Nullrate}
Vor der eigentlichen Messung wird die Nullrate bestimmt. Um den statistischen Fehler möglichst gering zu halten,
wird für eine Zeit von $t=500\,\unit{\second}$ gemessen. Bei dieser Messzeit ergibt sich eine Zählrate von
$N=150$. Die Nullrate bestimmt sich dadurch zu $N_{\symup{u}}=3$ pro $10\,\unit{\second}$ oder zu
$N_{\symup{u}}=9$ pro $30\,\unit{\second}$. Im Folgenden wird die Nullrate immer von den Messwerten abgezogen.

\section{Halbwertszeit von Vanadium}
\label{sec:Vanadium}
Um die Halbwertszeit von Vanadium zu bestimmen, wird die mittlere Zählrate halblogarithmisch gegen die Zeit
in \autoref{fig:vanadium} aufgetragen. Die gemessen Werte samt des $\sqrt{N}$-Fehlers sind in \autoref{tab:vanadium}
zu finden.
\begin{table}
  \centering
  \begin{tabular}{c c}
    \toprule
    Messzeit $t/\unit{\second}$ & Zählrate $N$\\
    \midrule
     30 & 131 $\pm$ 11 \\
     60 & 116 $\pm$ 11 \\
     90 & 122 $\pm$ 11 \\
    120 &  88 $\pm$  9 \\
    150 &  91 $\pm$ 10 \\
    180 &  72 $\pm$  1 \\
    210 &  81 $\pm$  8 \\
    240 &  63 $\pm$  9 \\
    270 &  58 $\pm$  8 \\
    300 &  40 $\pm$  8 \\
    330 &  57 $\pm$  6 \\
    360 &  36 $\pm$  8 \\
    390 &  40 $\pm$  6 \\
    420 &  40 $\pm$  6 \\
    450 &  49 $\pm$  6 \\
    \bottomrule
  \end{tabular}
  \begin{tabular}{c c}
    \toprule
    Messzeit $t/\unit{\second}$ & Zählrate $N$\\
    \midrule
    480 &  28 $\pm$ 7 \\
    510 &  19 $\pm$ 5 \\
    540 &  29 $\pm$ 4 \\
    570 &  28 $\pm$ 5 \\
    600 &  19 $\pm$ 5 \\
    630 &   9 $\pm$ 4 \\
    660 &  22 $\pm$ 3 \\
    690 &  13 $\pm$ 5 \\
    720 &  12 $\pm$ 4 \\
    750 &  12 $\pm$ 3 \\
    780 &  12 $\pm$ 3 \\
    810 &  12 $\pm$ 3 \\
    840 &   6 $\pm$ 2 \\
    870 &   9 $\pm$ 3 \\
    900 &   5 $\pm$ 2 \\
    \bottomrule
  \end{tabular}
  \caption{Messwerte der Zählrate für Vanadium.}
  \label{tab:vanadium}
\end{table}
Der Fehler für die gemittelte Zählrate ergibt sich durch die Gaußsche Fehlerforpflanzung, die durch
\begin{align*}
  \Delta\bar{N} = \frac{1}{t}\cdot\Delta N
\end{align*}
gegeben ist. Mit Hilfe von \autoref{eqn:Zerfallsgesetz} lässt sich in halblogarithmischer Darstellung die
Gleichung
\begin{align*}
  ln(N(t)) = -\lambda t + \ln{N_{\symup{0}}}
\end{align*}
aufstellen. Mit Scipy lässt sich nun eine lineare Regression mit den Ergebnissen
\begin{align*}
  \lambda_{\symup{Vanadium}} &= (6,40 \pm 0,28)*10^{-3} \frac{1}{\unit{\second}} \\
  ln(N_{\symup{0, Vanadium}}) &= 0,49 \pm 0,15 \\
\end{align*}
durchführen. Aus \autoref{eqn:T} ergibt sich dann für die Halbwertszeit von Vanadium
\begin{align*}
  T_{\symup{Vanadium}} = (108 \pm 5)\,\unit{\second}
\end{align*}

\begin{figure}
  \centering
  \includegraphics{plot1.pdf}
  \caption{Messwerte und lineare Regression für Vanadium.}
  \label{fig:vanadium}
\end{figure}

\subsection{Halbwertszeiten der Zerfallsprodukte von Silber}
\label{sec:Silber}
Die Messwerte einschließlich des $\sqrt{N}$-Fehlers für das Silberpräparat sind in \autoref{tab:silber} zu finden.
Es wurde über einen Zeitraum von $420\,\unit{s}$ in $10\,\unit{\second}$ Intervallen gemessen.
\begin{table}
  \centering
  \begin{tabular}{c c}
    \toprule
    Messzeit $t/\unit{\second}$ & Zählrate $N$ \\
    \midrule
     10 & 156 $\pm$ 12 \\
     20 & 143 $\pm$ 12 \\
     30 & 106 $\pm$ 10 \\
     40 &  85 $\pm$  9 \\
     50 &  67 $\pm$  8 \\
     60 &  42 $\pm$  6 \\
     70 &  57 $\pm$  7 \\
     80 &  36 $\pm$  6 \\
     90 &  42 $\pm$  6 \\
    100 &  30 $\pm$  5 \\
    110 &  29 $\pm$  5 \\
    120 &  21 $\pm$  4 \\
    130 &  28 $\pm$  5 \\
    140 &  21 $\pm$  4 \\
    150 &  34 $\pm$  6 \\
    160 &  10 $\pm$  3 \\
    170 &  14 $\pm$  3 \\
    180 &  17 $\pm$  4 \\
    190 &  13 $\pm$  3 \\
    200 &  13 $\pm$  3 \\
    210 &  19 $\pm$  4 \\
    \bottomrule
  \end{tabular}
  \begin{tabular}{c c}
    \toprule
    Messzeit $t/\unit{\second}$ & Zählrate $N$ \\
    \midrule
    220 & 18 $\pm$ 4 \\
    230 & 13 $\pm$ 3 \\
    240 &  9 $\pm$ 2 \\
    250 & 23 $\pm$ 4 \\
    260 & 12 $\pm$ 3 \\
    270 &  9 $\pm$ 2 \\
    280 & 11 $\pm$ 3 \\
    290 & 11 $\pm$ 3 \\
    300 & 16 $\pm$ 4 \\
    310 &  9 $\pm$ 2 \\
    320 &  9 $\pm$ 2 \\
    330 & 20 $\pm$ 4 \\
    340 &  6 $\pm$ 2 \\
    350 &  6 $\pm$ 2 \\
    360 & 10 $\pm$ 3 \\
    370 &  4 $\pm$ 1 \\
    380 & 21 $\pm$ 4 \\
    390 & 16 $\pm$ 4 \\
    400 & 12 $\pm$ 3 \\
    410 &  6 $\pm$ 2 \\
    420 & 11 $\pm$ 3 \\
    \bottomrule
  \end{tabular}
  \caption{Messwerte der Zählrate für Silber}
  \label{tab:silber}
\end{table}
Um die Halbwertszeiten von $\ce{^108 Ag}$ und $\ce{^110 Ag}$ zu bestimmen, wird in \autoref{fig:silber} die gemittelte Zählrate
halblogarithmisch gegen die Zeit aufgetragen. Um die beiden Zerfälle der Isotope $\ce{^108 Ag}$ und $\ce{^110 Ag}$
zu trennen wird die Messung an einer Stelle $t^{*} = 120 \,\unit{\second}$ in zwei Bereiche aufgeteilt. Links von
$t^{*}$ sind noch beide Zerfälle vorhanden. Rechts davon nur noch der langlebige Zerfall des Isotops $\ce{^108 Ag}$.
\begin{figure}
  \centering
  \includegraphics{plot2.pdf}
  \caption{Messwerte und zweifache lineare Regression für Silber.}
  \label{fig:silber}
\end{figure}

\subsubsection{Halbwertszeit von $\ce{^108 Ag}$}
\label{sec:108Ag}
Aus der linearen Regression ergeben sich für den rechten Bereich
\begin{align*}
  \lambda_{\symup{\ce{^108 Ag}}}  &= (80,95 \pm 13,46)\cdot 10^{-4}\,\frac{1}{\unit{\second}} \\
  ln(N_{\symup{0, \ce{^108 Ag}}}) &= 2,54 \pm 0,23. \\
\end{align*}
Aus \autoref{eqn:T} ergibt sich für die Halbwertszeit von $\ce{^108 Ag}$
\begin{align*}
  T_{\symup{\ce{^108 Ag}}} = (86 \pm 14)\,\unit{\second} 
\end{align*}

\subsubsection{Halbwertszeit von $\ce{^110 Ag}$}
\label{sec:110Ag}
Da $\ce{^110 Ag}$ schneller zerfällt, muss die theoretisch ermittelte Zählrate von $\ce{^108 Ag}$ von den Messwerten
abgezogen werden. Die abzuziehende Zählrate lässt sich durch \autoref{eqn:lnDelta} bestimmen. Die korrigierten Werte
für die lineare Regression ergeben sich aus
\begin{align*}
  N_{\symup{kurz}}(t) &= N_{\symup{Ges}}(t) - N_{\symup{lang}}(t) \\
                      &= N_{\symup{Ges}}(t) - e^{-mt + b}
\end{align*}
Für die lineare Regression ergeben sich die Werte
\begin{align*}
  \lambda_{\symup{\ce{^110 Ag}}}  &= (89,67 \pm 18,04)\cdot 10^{-3}\,\frac{1}{\unit{\second}} \\
  ln(N_{\symup{0, \ce{^110 Ag}}}) &= 3,59 \pm 0,81. \\
\end{align*}
wobei diesmal nur $t$ von $0\,\unit{\second}$ bis $80\,\unit{\second}$ betrachtet wird.
Daraus ergibt sich eine Halbwertszeit von
\begin{align*}
  T_{\symup{\ce{^110 Ag}}} = (7,7 \pm 1,6)\,\unit{\second}.
\end{align*}
Der Fehler bestimmt sich durch die Gaußsche Fehlerforpflanzung durch
\begin{equation*}
  \Delta T = \frac{\sqrt{ln(2)}}{\lambda} \cdot \Delta \lambda\,. 
\end{equation*}