\section{Diskussion}
\label{sec:Diskussion}
Insgesamt haben sich für vier der fünf Kennlinien Kurvenverläufe eingestellt aus denen der Sättigungsstrom mit
einer hohen Genauigkeit ablesbar ist. Nur die Kennlinie für $I_{\symup{H}}=\SI{2,5}{\ampere}$ hat ihren
Sättigungswert erreicht, so dass der maximale Wert als Sättigungswert stark fehlerbehaftet ist und bei weiteren
Berechnung die Genauigkeit der Ergebnisse verringert. Für den Exponenten aus dem Langmuir-Schottkyschen
Raumladungsgesetzes wurde ein experimenteller Wert von $m = 1,439 \pm 0008$ bestimmt, der eine relative Abweichung
zum Literaturwert $m=\frac{3}{2}$ von $4,06\,\%$ aufweist. Mögliche Fehlerquellen sind hier, dass der
Raumladungsbereich nur geschätzt werden kann. Dennoch ist die Abweichung so gering, dass sich dadurch die
Gültigkeit des Langmuir-Schottkyschen Raumladungsgesetzes zeigen lässt. Weiterhin wurde noch die Kathodentemperatur
für die Kennlinie mit $I_{\symup{H}}=\SI{2,5}{\ampere}$ aus den Messwerten des Anlaufstroms zu
$T = (1700 \pm 90)\si{\kelvin}$ bestimmt. Dieser Wert lässt sich mit dem aus der Leistungsbilanz berechnetem Wert
von $2248\,\si{\kelvin}$ vergleichen, so dass sich eine relative Abweichung von $24,3\,\%$ trotz entsprechender
Korrektur der Messwerte ergibt. Auffällig ist dabei der letzte Wert, der stark von dem von einer Gerade zu
erwartenen Verlauf abweicht. Würde dieser Punkt nicht berücksichtig werden, würde das aus der linearen Regression
bestimmte $m$ kleiner werden und sich so ein Wert mit geringerer Abweichung ergeben. Außerdem wurde noch die
Austrittsarbeit der Elektronen aus Wolfram zu $\bar{\phi} = (4,50 \pm 0,31)\si{\eV}$ bestimmt. Der Literaturwert
ist durch $\bar{\phi}_{\symup{Lit}} = \SI{4,54}{\eV}$ gegeben. Woraus sich eine relative Abweichung von
$0,88\,\%$ ergibt. Da diese Abweichungen äußerst gering ist, lässt sich sagen, dass die meisten theoretischen
Werte durch das Experiment bestätigt wurden. 