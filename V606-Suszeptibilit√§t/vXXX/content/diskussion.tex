\section{Diskussion}
\label{sec:Diskussion}

\subsection{Filterkurve des Selektiv-Verstärkers}

Bei der experimentellen Messung der Güte des Filters ergab sich ein Wert
von $Q = (13, 37 \pm 0, 04)$. Dieser weicht von dem Theoriewert $Q = 20$
um $33,15 \%$ ab. Der Fit aus der sich die berechnete Güte ergibt, schmiegt
sich sehr gut an die Messwerte im Bereich um das Maximum herum an. Die 
Glockenkurve sinkt allerdings schneller als die Messwerte, was die Ungenauigkeit
des Ergebnis erklären könnte. Eine weitere Fehlerquelle ist das Ablesen der Messwerte.
Das Voltmeter zeigt während des Versuchs stark schwankende Werte bei gleichbleibender
Frequenz an.

\subsection{Bestimmung der Suszeptibilitäten drei seltener Erden}

Die Abweichungen der experimentellen Ergebnisse zu den Theoretischen Werten 
werden jeweils mit beiden Methoden separat berechnet und lassen sich in \autoref{tab:AbwWid}
und \autoref{tab:AbwSp} finden.
\begin{table}
    \centering
    \caption{Abweichungen der Theoriewerte der Suszeptibilitäten von den experimentellen Ergebnissen aus den Widerständen.}
    \label{tab:AbwWid}
    \begin{tabular}{c || c | c | c }
      \toprule
      Probenname & $\chi_{\mathrm{theo}}$ & $\overline{\chi}_{\mathrm{exp}}$ & Abweichung \\
      \hline
      $\mathrm{Nd}_2 \mathrm{O}_3$ & 0,0030 & 0,0033 $\pm$ 0,0006   & 10,84 \%\\
      $\mathrm{Gd}_2 \mathrm{O}_3$ & 0,0138 & 0,0131 $\pm$ 0        & 4,86 \%\\
      $\mathrm{Dy}_2 \mathrm{O}_3$ & 0,0254 & 0,0233 $\pm$ 0,0007   & 8,30 \%\\
      \midrule
      \bottomrule
    \end{tabular}
\end{table}
  
\begin{table}
    \centering
    \caption{Abweichungen der Theoriewerte der Suszeptibilitäten von den experimentellen Ergebnissen aus den Spannungen.}
    \label{tab:AbwSp}
    \begin{tabular}{c || c | c | c }
      \toprule
      Probenname & $\chi_{\mathrm{theo}}$ & $\overline{\chi}_{\mathrm{exp}}$ & Abweichung \\
      \hline
      $\mathrm{Nd}_2 \mathrm{O}_3$ & 0,0030 & 0,0194 $\pm$ 0,0012   & 542,54 \%\\
      $\mathrm{Gd}_2 \mathrm{O}_3$ & 0,0138 & 0,0067 $\pm$ 0        & 51,43 \%\\
      $\mathrm{Dy}_2 \mathrm{O}_3$ & 0,0254 & 0,0430 $\pm$ 0,006    & 69,71 \%\\
      \midrule
      \bottomrule
    \end{tabular}
\end{table}

Durch die sehr hohen Abweichungen der Bestimmung der Suszeptibilität durch die Spannungen
kann man diese Messreihe nicht als vertrauenswürdig ansehen. Hier muss ein Fehler beim Messen 
aufgetreten sein. Da hier die Spannung am stark schwankenden Voltmeter abgelesen wurde,
kann dies erneut als Fehlerquelle betrachtet werden.\\
Die Messreihe mit den Widerständen hat nur geringe Abweichungen mit den Theoriewerten.
Durch diese Messungen kann die Theorie bestätigt werden und das Messverfahren mit den Widerständen als 
bevorzugte Methode identifiziert werden.