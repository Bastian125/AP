\section{Auswertung}
\label{sec:Auswertung}

\subsection{Bestimmung der Apparatenkonstante}
\label{sec:Apparatenkonstante}
Um die Apparatenkonstante zu bestimmen, werden die Dichten der Kugeln benötigt.
Die Messwerte der Massen und Radien der beiden Kugeln sind in Tabelle \ref{tab:MasseundDichte} zu finden.
Die Radien und Massen bestimmen sich zu
\begin{align*}
  r_{\symup{Gr}} &= (7,86 \pm 0,04)\cdot 10^{-3} \,\si{\meter}\\
  m_{\symup{Gr}} &= (4,54 \pm 0,01) \cdot 10^{-3} \,\si{\kilo\gram}\\
  r_{\symup{Kl}} &= (7,64 \pm 0,11)\cdot 10^{-3} \,\si{\meter}\\
  m_{\symup{Kl}} &= (4,44 \pm 0,02) \cdot 10^{-3} \,\si{\kilo\gram}.\\
\end{align*}

Aus () ergibt sich für die Dichten
\begin{align*}
  \rho_{\symup{Gr}} &= (2232 \pm 34) \,\frac{\si{\kg}}{\si{m^3}}  \\
  \rho_{\symup{Kl}} &= (2380 \pm 10) \,\frac{\si{\kg}}{\si{m^3}}. \\
\end{align*}

\begin{table}
  \centering
  \caption{Messdaten der Massen und Radien der beiden Kugeln.}
  \label{tab:MasseundDichte}
  \begin{tabular}{c c c c}
    \toprule
    $r_{\symup{Gr}}/10^{-3}\,\unit{\meter}$ & $m_{\symup{Gr}}/10^{-3}\,\unit{\kilo\gram}$ & $r_{\symup{Kl}}/10^{-3}\,\unit{\meter}$ & $m_{\symup{Kl}}/10^{-3}\,\unit{\kilo\gram}$ \\
    \midrule
    7,90 & 4,54 & 7,75 & 4,46 \\
    7,80 & 4,56 & 7,55 & 4,46 \\
    7,85 & 4,54 & 7,55 & 4,43 \\
    7,90 & 4,54 & 7,80 & 4,42 \\
    7,85 & 4,54 & 7,55 & 4,43 \\
    \bottomrule
  \end{tabular}
\end{table}
Weiterhin werden noch die Fallzeiten der beiden Kugeln im Viskosimeter benötigt.
Die Fallzeiten der kleinen und der großen Kugel sind in Tabelle \ref{tab:Fallzeit} angegeben. Aus den Messwerten ergibt sich für die Fallzeiten
\begin{align*}
  t_{\symup{Kl}} &= (12,04 \pm 0,18)\,\unit{\second} \\
  t_{\symup{Gr}} &= (39,7 \pm 0,39)\,\unit{\second}. \\
\end{align*}

\begin{table}
  \centering
  \caption{Fallzeiten der Kugeln.}
  \label{tab:Fallzeit}
  \begin{tabular}{c c | c c}
    \toprule
    $t_{\symup{Kl}}/\unit{\second} $ & $t_{\symup{Gr}}/\unit{\second}$ & $t_{\symup{Kl}}/\unit{\second} $ & $t_{\symup{Gr}}/\unit{\second}$ \\
    \midrule
    12,28 & 40,17 & 12,12 & 39,52 \\
    12,06 & 40,17 & 12,13 & 39,60 \\
    11,94 & 39,27 & 11,88 & 39,43 \\
    11,97 & 39,67 & 11,97 & 39,49 \\
    12,12 & 39,67 & 12,09 & 38,92 \\
    12,12 & 39,55 & 12,47 & 39,09 \\
    12,13 & 39,40 & 12,15 & 40,11 \\
    11,87 & 39,57 & 12,06 & 40,17 \\
    11,87 & 40,52 & 12,06 & 39,99 \\
    11,59 & 39,85 & 11,84 & 39,88 \\
    \bottomrule
  \end{tabular}
\end{table}

Nun wird mit Hilfe der Fallzeitwerte der kleinen Kugel die Viskosität von Wasser bei Raumtemperatur bestimmt. Dabei werden
\begin{align*}
  K_{\symup{Kl}} &= 7,64\cdot 10^{-8} \,\frac{\unit{\pascal}\cdot\unit{\meter^3}}{\unit{\kilo\gram}} \\
  \rho_{\symup{Fl}} &= 998,2\,\frac{\unit{\kilo\gram}}{\unit{\meter^3}}
\end{align*}
verwendet. $K_{\symup{Kl}}$ ist die angegeben Appartenkonstante für die kleine Kugel \cite{anleitung107}  und $\rho_{\symup{Fl}}$
die Dichte von Wasser bei Raumtemperatur \cite{waterdensity}. Aus (\ref{eqn:etaK}) ergibt sich für die Viskosität des Wassers bei Raumtemperatur
\begin{align*}
  \eta_{\symup{RT}} = (1,27 \pm 0,02)\cdot 10^{-3} \,\unit{\pascal}\cdot\unit{\second}.
\end{align*}
Durch Umstellen von (\ref{eqn:etaK}) und einsetzen der Werte ergibt sich nun
\begin{align*}
  K_{\symup{Gr}} = (2,59 \pm 0,09)\cdot 10^{-8}\,\frac{\unit{\pascal}\cdot\unit{\meter^3}}{\unit{\kilo\gram}}
\end{align*}
für die Apparatenkonstante der großen Kugel.

\subsection{Temperaturabhängigkeit der Viskosität}
\label{sec:TempVisk}
Nun kann mit (\ref{eqn:etaK}) die Viskosität für unterschiedliche Temperaturen berechnnet werden. Die Werte für die Fallzeit der Kugel und Temperatur und Dichte \cite{waterdensity}
 des Wassers sind
in Tabell \ref{tab:Temp} angegeben.

\begin{table}
  \centering
  \caption{Messdaten zur Bestimmung der Temperaturabhängigkeit der Viskosität.}
  \label{tab:Temp}
  \begin{tabular}{c c c c | c c c c}
    \toprule
    $T/\unit{\kelvin}$ & $t/\unit{\second}$ & $\rho_{\symup{Fl}}/\frac{\unit{\kilo\gram}}{\unit{\meter^3}}$ & $\eta/10^{-3}\unit{\pascal}\cdot \unit{\second}$ & $T/\unit{\kelvin}$ & $t/\unit{\second}$ & $\rho_{\symup{Fl}}/\frac{\unit{\kilo\gram}}{\unit{\meter^3}}$ & $\eta/10^{-3}\unit{\pascal}\cdot \unit{\second}$ \\
    \midrule
    293,15 & 39,90 & 998,21 & 1,276 & 308,15 & 28,33 & 994,03 & 0,909 \\
    293,15 & 39,27 & 998,21 & 1,256 & 308,15 & 28,22 & 994,03 & 0,906 \\
    293,15 & 39,61 & 998,21 & 1,267 & 311,15 & 27,77 & 992,97 & 0,892 \\
    293,15 & 38,93 & 998,21 & 1,245 & 311,15 & 27,43 & 992,97 & 0,881 \\
    296,15 & 37,92 & 997,54 & 1,214 & 311,15 & 27,76 & 992,97 & 0,892 \\
    296,15 & 36,39 & 997,54 & 1,165 & 311,15 & 27,35 & 992,97 & 0,879 \\
    296,15 & 36,94 & 997,54 & 1,182 & 314,15 & 26,48 & 991,83 & 0,851 \\
    296,15 & 36,61 & 997,54 & 1,172 & 314,15 & 25,02 & 991,83 & 0,805 \\
    299,15 & 35,61 & 996,79 & 1,140 & 314,15 & 26,09 & 991,83 & 0,839 \\
    299,15 & 34,52 & 996,79 & 1,106 & 314,15 & 25,36 & 991,83 & 0,815 \\
    299,15 & 35,11 & 996,79 & 1,124 & 317,15 & 24,61 & 990,63 & 0,792 \\
    299,15 & 34,49 & 996,79 & 1,105 & 317,15 & 24,40 & 990,63 & 0,785 \\
    302,15 & 32,74 & 995,95 & 1,049 & 317,15 & 24,25 & 990,63 & 0,781 \\
    302,15 & 31,61 & 995,95 & 1,013 & 317,15 & 24,38 & 990,63 & 0,785 \\
    302,15 & 32,33 & 995,95 & 1,036 & 320,15 & 23,42 & 989,37 & 0,755 \\
    302,15 & 31,83 & 995,95 & 1,020 & 320,15 & 22,80 & 989,37 & 0,735 \\
    305,15 & 30,80 & 995,03 & 0,988 & 320,15 & 23,05 & 989,37 & 0,743 \\
    305,15 & 30,27 & 995,03 & 0,971 & 320,15 & 22,95 & 989,37 & 0,739 \\
    305,15 & 30,52 & 995,03 & 0,979 & 323,15 & 21,55 & 988,04 & 0,695 \\
    305,15 & 30,22 & 995,03 & 0,969 & 323,15 & 21,71 & 988,04 & 0,700 \\
    308,15 & 28,33 & 994,03 & 0,909 & 323,15 & 21,58 & 988,04 & 0,696 \\
    308,15 & 28,80 & 994,03 & 0,924 & 323,15 & 22,08 & 988,04 & 0,712 \\
    \bottomrule
  \end{tabular}
\end{table}