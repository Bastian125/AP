\section{Durchführung}
\label{sec:Durchfuehrung}

Zunächst wird mit Hilfe einer Schieblehre und einer Waage die Dichte einer kleinen und einer großen Kugel bestimmt. Daraufhin wird mit einer
Libelle überprüft, ob das Viskosimeter gerade steht und nachjustiert.\\
Sobald Kugeln vermessen und der Versuchsaufbau überprüft wurde, wird das Fallrohr mit destillierten Wasser aufgefüllt und mit einem Glasstab darauf 
geachtet, dass sich keine Blasen bilden.\\
Bei Raumtemperatur wird nun die große Kugel in das Rohr gegeben und die das Viskosimeter von beiden Seiten dicht verschlossen. Mit Hilfe einer Stoppuhr
wird nun die Fallzeit der Kugel zwischen den Markierungen gemessen. Wenn die untere Marke von der Kugel passiert wurde, wird das Viskosimeter um $\mathrm{180^{\circ}}$
gedreht und die Messung wiederholt. Insgesamt werden 20 Messwerte aufgenommen. Dasselbe wird mit der kleinen Kugel gemacht.\\
Mithilfe der Messdaten wird nun die Apparaturkonstante $K_{gr}$ berechnet. Die Apparaturkonstante für die kleine Kugel ist gegeben mit $K_{kl} = 0.07640 \, \mathrm{m\,Pa\,cm^3/g}$.\\
Anschließend wird der gleiche Versuch mit verschiedenen Temperaturen durchgeführt. Die erste Messung findet bei $\mathrm{20\,{^\circ C}}$ statt. Es werden jeweils 4 Messungen pro
Temperatur durchgeführt. Die Temperatur wird bei jedem Durchgang mit dem Thermostat um jeweils $\mathrm{3\,{^\circ C}}$ erhöht. Nach jeder Temperaturerhöhung müssen
einige Minuten abgewartet werden, sodass sich das destillierte Wasser in dem Viskosimeter ebenfalls erwärmt. Am Ende hat man nun 44 Messwerte.\\
Mithilfe der Messwerte kann nun die dynamische Viskosität $\eta(T)$ des Wassers und die Reynoldzahl bestimmt werden.

\newpage