\section{Diskussion}
\label{sec:Diskussion}
Insgesamt haben sich Werte zwischen $0,4$ bis ca. $1\,\unit{\volt}$ bestimmen lassen, die im realistischen
Bereich für die Grenzspannungen liegen. Das Plancksche Wirkungsquantum durch die Elementarladung wurden
zu $\frac{h}{e_{\symup{0}}}=2,4\cdot10^{-15}\,\unit{\volt\second}$ bestimmt. Dadurch ergbit sich eine Abweichung
zum Theoriewert $\frac{h}{e_{\symup{0}}}=4,14\cdot 10^{-15}\,\unit{\volt\second}$ von $72,5\%$. Die Abweichung ist
relativ hoch und lässt sich auf den beschädigten Aufbau zurückführen. So war der Auslenkarm der Photozelle nicht
orthogonal zum Tisch, so dass das Intensitätsmaximum der Spektrallinien nicht auf die Photozelle traf. Dadurch
sind alle Werte zu klein. Obwohl die quantitativen Aussagen nur grob überprüfbar waren, sind die qualitativen
Zusammenhänge des Photoeffekts eindeutig zeigbar.