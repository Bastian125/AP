\section{Auswertung}
\label{sec:Auswertung}
Es werden die Bremsspannungen $U_{\symup{B}}$ gegen den Photostrom $I$ für die rote, gelbe, grüne
und die beiden violetten Spektrallinien aufgetragen und mit Hilfe einer durch Python berechneten
Ausgleichsgeraden die Grenzspannung $U_G$ bestimmt.

\subsection{Rote Spektrallinie}
\label{sec:rote_Spektrallinie}
Die Messwerte sind in \autoref{tab:rot} und die Ausgleichsgerade in \autoref{fig:rot} zu finden.
\begin{table}
    \centering
    \caption{Messwerte für die rote Spektrallinie.}
    \label{tab:rot}
    \begin{tabular}{c c}
        \toprule
        $U_{\symup{B}} / \unit{\milli\volt}$ & $I / \unit{\nano\ampere}$ \\
        \midrule
          0 & 0.040 \\
         20 & 0.030 \\
         40 & 0.025 \\
         60 & 0.020 \\
         80 & 0.020 \\
        100 & 0.020 \\
        120 & 0.020 \\
        220 & 0.010 \\
        380 & 0.000 \\
        \bottomrule
    \end{tabular}
\end{table}

\begin{figure}
    \centering
    \label{fig:rot}
    \caption{Photostrom in Abhängigkeit der Bremsspannung und Ausgleichsgerade für die rote Spektrallinie.}
    \includegraphics[width=\textwidth]{plot1.pdf}
\end{figure}