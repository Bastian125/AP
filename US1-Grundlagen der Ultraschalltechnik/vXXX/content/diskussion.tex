\newpage
\section{Diskussion}
\label{sec:Diskussion}

\subsection{Wellenlänge und Frequenz einer $2 \, \si{\mega\hertz}$ Schwingung}

Es wurde eine $2 \, \si{\mega\hertz}$ Schwingung in der Geräteeinstellung der Sonde 
eingestellt und mittels des Visualisierungsprogramms die Länge einer Periode vermessen
und damit die Wellenlänge und Frequenz der Schwingung bestimmt.\\
Dieses Verfahren führte zu einer Frequenz von $f = 2 272 727,27 \, \si{\hertz}$ und einer
Wellenlänge von $\lambda = (1,19 \pm 0,01) \cdot 10^{-3} \, \si{\meter}$.\\
Zuvor wurde mittels Literaturwerte durch dieselben Gleichungen ebenfalls die Frequenz
und Wellenlänge einer $2 \, \si{\mega\hertz}$ Schwingung ausgerechnet. Dies ergab eine
Frequenz von $f_{\mathrm{lit}} = 2 000 000,00 \, \si{\hertz}$ und eine Wellenlänge von
$\lambda_{\mathrm{lit}} = 1,365 \cdot 10^{-3} \, \si{\m}$.\\
Wenn man nun Literaturwert und Messergebnis der Wellenlänge vergleicht, erhält man eine 
Abweichung von 14,71 \%.\\
Die Messung der fünf Perioden im Programm wurde visuell nach Augenmaß durchgeführt und
erzeugt somit eine unbekannte Messungenauigkeit. Daher ist eine Abweichung von 14,71 \%
in einem akzeptablen Bereich und verifiziert die Messmethode.\\

\subsection{Messungen der Schallgeschwindigkeit in Acryl}

Um die Schallgeschwindigkeit in Acryl zu messen, wurden drei Versuchsreihen durchgeführt.\\
Zweimal wurde das Echo-Impuls-Verfahren benutzt, einmal bei einem Acrylblock und einmal
bei mehreren Acrylzylindern. Die Acrylzylinder wurden dann erneut für die Durchschallungs-Messmethode
verwendet.\\
Der Acrylblock ergab eine Geschwindigkeit von $c_{\mathrm{Block}} = (2714,87 \pm 13,64) \, \si{\meter\per\second}$,
die Acrylzylinder mit Echo-Impuls-Verfahren $c_{\mathrm{Zyl-EcIm}} = (2758,95 \pm 16,31) \, \si{\meter\per\second}$
und die Acrylzylinder mit Durchschallungs-Verfahren $c_{\mathrm{Zyl-Durch}} = (2759,13 \pm 71,81) \, \si{\meter\per\second}$.
Die drei Messergebnisse haben Insgesamt eine mittlere Abweichung von 1,09 \%.\\
Da diese Abweichung sehr gering ist, lässt sich die Theorie durch die Versuchsreihe bestätigen.\\

\subsection{Messungen der Dämpfung in Acryl}

Die Messergebnisse liefern eine Dämpfung von $\alpha = (23,2 \pm 3,2) \, \si{\per\meter}$. In diesem
Ergebnis ist ein Messergebnis nicht mit einbezogen worden, da dieses starke Abweichungen von den anderen
aufzeigte. Hätte man das Messergebnis mit einbezogen, wäre die Messungenauigkeit sehr hoch gewesen. Wahrscheinlich
sogar höher als das $\alpha$ selbst.\\
Dieser fehlerhafte Messwert lässt sich entweder durch Beschädigungen im Acrylglas oder durch einen
systematischen Fehler bei der Messung selbst, durch falsches Messen erklären.\\
Eine Bestätigung der Dämpfung würde eine neue Messreihe ohne fehlerhafte Werte erfordern, kann jedoch
im Rahmen des Anfängerpraktikums als akzeptabel angenommen werden.\\


