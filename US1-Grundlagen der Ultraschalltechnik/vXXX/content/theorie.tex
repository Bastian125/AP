\section{Zielsetzung}
\label{sec:Zielsetzung}
In diesem Versuch soll die Funktionsweise der Ultraschalltechnik untersucht werden, da sie vorallem in der
medizinischen und materialwissenschaftlichen Technik ihre Anwendung findet und deshalb für die
Experimentalphysik von Relevanz ist.

\section{Theorie}
\label{sec:Theorie}
Den Frequqenzbereich des Schalls lässt sich in vier Teilbereiche aufteilen. Einmal in den für das menschliche
Ohr hörbare Bereich von ca. $16\,\unit{\hertz}$ bis ca. $20\,\unit{\kilo\hertz}$, in den Teilbereich 
von ca. $20\,\unit{\kilo\hertz}$ bis ca. $1\,\unit{\giga\hertz}$, der auch Ultraschall genannt wird und in den
Teilbereich ab $1\,\unit{\giga\hertz}$, der auch Hyperschall genannt wird. Der Teilbereich unter dem
hörbaren Bereich wird auch Infraschall genannt.

\subsection{Ausbreitung von Schallwellen}
\label{sec:Schallwellen}
Bei Schallwellen handelt es sich um longitudinale Wellen, die sich aufgrund von Druckschwankungen fortbewegt.
Eine Schallwelle mit Ausbreitung in x-Richtung lässt sich mit
\begin{equation*}
    p(x, t) = p_0 + v_0 Z \cos{(\omega t - kx)}
\end{equation*}
darstellen.\\
Dabei ist $p_0$ der Normaldruck, $v_0$ die Schallschnelle, und $Z=\rho c$ die Akustische
Impedanz mit der Schallgeschwindigkeit $c$ und der Dichte $\rho$ des durchschallten Materials.\\
Die Schallgeschwindigkeit in Flüssigkeiten lässt sich in Abhängigkeit ihrer Kompressibilität $\kappa$ und ihrer
Dichte $\rho$ durch
\begin{equation*}
    c_{\mathrm{Fl}} = \sqrt{\frac{1}{\kappa \cdot \rho}}
\end{equation*}
ausdrücken. Da in Festkörpern auch die transversalen Wellen aufgrund der Schubspannung mit berücksichtigt werden
müssen, wird $\kappa$ durch den Zusammenhang zum Elastizitätsmodul $E$ ersetzt. Es ergibt sich
\begin{equation*}
    c_{\mathrm{Fe}} = \sqrt{\frac{E}{\rho}}.
\end{equation*}
Im Normalfall geht immer eine Teil der Energie bei der Schallausbreitung durch Absorption verloren. Daraus bestimmt
sich die Intensität zu
\begin{equation}
    \label{eqn:Dämpfung}
    I(x) = I_0 \cdot e^{-\alpha x}.
\end{equation}
Dabei ist $\alpha$ der Absorptionskoeffizient der Schallamplitude und $I_0$ die Intensität des ausgehenden 
Schallimpulses. Um den Absorptionskoeffizienten möglichst
gering zu halten, wird aufgrund des hohen Absorptionskoeffizienten der Luft ein Kontaktmittel zwischen Schallgeber
und Material verwendet.\\
\\
Eine für den Versuch wichtige Eigenschaft von Schallwellen ist, dass diese reflektiert werden können. Der Reflexionskoeffizient ergibt sich in Abhänngigkeit der
akustischen Impedanzen $Z_{\symup{1}}$ und $Z_{\symup{2}}$ der Materialien durch
\begin{equation*}
    R = \left(\frac{Z_{\symup{1}}- Z_{\symup{2}}}{Z_{\symup{1}}+Z_{\symup{2}}}\right)^2 .
\end{equation*}
Für den Transmissionskoeffizienten ergibt sich
\begin{equation*}
    T = 1 - R.
\end{equation*}

\subsection{Erzeugung von Ultraschall}
\label{sec:Ultraschall}
Ultraschallwellen können durch den piezo-elektrischen Effekt erzeugt werden. Dazu wird ein piezo-elektrischer
Kristall, z.B. Quarz, in ein elektrisches Wechselfeld gebracht. Der Kristall gerät dadurch in Schwingung und
erzeugt Ultraschallwellen. Bei Resonanz werden große Ultraschallamplituden erreicht.\\
Umgekehrt kann bei angeregten Schwingungen ein elektrisches Wechselfeld gemessen werden, sodass die Quelle
der Ultraschallwellen auch als Empfänger dienen kann.

\subsection{Ultraschallverfahren}
\label{sec:Ultraschallverfahren}
Es werden in der Ultraschalltechnik zwei Verfahren verwand.
\begin{enumerate}
    \item $Durchschallungsverfahren$ - Beim Durchschallungsverfahren wird ein Sender und ein Empfänger verwendet. Dabei befindet sich die
    Probe ziwschen dem Sender und dem Empfänger. Wenn der Sender nun einen Schallimpuls aussendet, bewegt sich
    dieser durch die Probe und verliert an Intensität. Weiterhin kann durch Störstellen die Intensität auch
    verringert werden bevor der Impuls zum Zeitpunkt $t$ auf den Empfänger trifft.
    \item $Impuls$-$Echo$-$Verfahren$ - Beim Impuls-Echo-Verfahren wird der Sender gleichzeitig als Empfänger verwendet. Nach der Laufzeit
    $t$ wird die Intensität des Schallimpulses gemessen. Aus der Schallgeschwindigkeit $c$ und der Laufzeit $t$
    lässt sich der Abstand zur Störstelle in der Probe durch
    \begin{equation}
        \label{eqn:Strecke}
        s = \frac{ct}{2}
    \end{equation}
    bestimmen.
\end{enumerate}

\subsection{Scan-Methoden}
\label{sec:Scan}
Um die Messungen mit den zwei Ultraschalltechniken auszuwerten, gibt es drei unterschiedliche Verfahren.
\begin{enumerate}
    \item $A$-$Scan$ - Dabei wird nur die Amplitude angezeigt, so dass auch Abschwächungen messbar sind.
    \item $B$-$Scan$ - Durch ein zweidimensionales Graubild lässt sich die Intensität der zurücklaufenden
    Welle bestimmen.
    \item $TM$-$Scan$ - Mit Hilfe einer hohen Wiederholungsfrequenz kann ein Video ähnliches Bild erzeugt
    werden.
\end{enumerate}