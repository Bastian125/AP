\section{Zielsetzung}
\label{sec:Zielsetzung}
In diesem Versuch soll der Elastizitätsmodul durch Biegen von Stäben unterschiedlicher Metalle und Legierungen bestimmt werden.

\section{Theorie}
\label{sec:Theorie}
Der Elastizitätsmodul $E$ ist eine materialabhängige Konstante, die die Gestaltsänderung eines Körpers unter einer Normalpannung $\sigma$ beschreibt.
Durch die aus der Normalpannung entstehende Längenänderung $\symup{\Delta} L$ lässt sich mit dem Hookschen Gesetz der Zusammenhang
\begin{align}
    \label{eqn:Elasitizitaetsmodul}
    \sigma = E \cdot \frac{\symup{\Delta} L}{L}
\end{align}
aufstellen. Bei der Biegungen von Stäben wirkt ein äußeres Drehmoment, das die oberen Schichten des Stabs dehnt und die unteren Schichten staucht.
In der Mitte des Stabs befindet sich die so genannte neutrale Faser, die aufgrund des Kräftegleichgewichts in ihrer Länge unverändert bleibt.
Der Stab kann sich so weit biegen bis das innere und das äußere Drehmoment gleich sind. Die Drehmomente sind durch
\begin{align*}
    M_{\symup{F}} &= F(L-x)\\
    M_{\symup{\sigma}} &= \int_{Q} y\sigma(y)\symup{dq}
\end{align*}
gegeben. $Q$ ist dabei der Querschnitt des Stabs und $y$ der Abstand des Flächenelements $\symup{dq}$ zur neutralen Faser.
Durch längere mathematische Vorüberlegungen ergibt sich für einen einseitig eingespannten Stab für die Durchbiegung
\begin{align}
    \label{eqn:Durchbiegung}
    D(x) = \frac{F}{2EI} \cdot\left(Lx^2 - \frac{x^3}{3}\right).
\end{align}
Dabei ist $x$ die Entfernung des Messpunktes zum Einspannpunkt, $I$ das Flächenträgheitsmoment und $L$ die Länge des Stabs.
Wenn der Stab beidseitig eingespannt ist und die Kraft in der Mitte des Stabs wirkt, ergibt sich für $0 \leq x \leq L/2$
\begin{align}
    \label{eqn:DurchbiegungL/2}
    D(x) = \frac{F}{48EI}\cdot \left(3L^2 x - 4x^3\right).
\end{align}
Für $L/2 \leq x \leq L$ ergibt sich
\begin{align}
    \label{eqn:DurchbiegungL}
    D(x) = \frac{F}{48EI}\cdot \left(4x^3 -12Lx^2 +9L^2 x - L^3\right).
\end{align}
Das Flächenträgheitsmoment eines Stabs mit quadratischen Querschnitt und der Seitenlänge $a$ ist durch
\begin{align}
    \label{eqn:quadrat}
    I_{\symup{\square}} = \frac{a^4}{12}
\end{align}
gegeben \cite{flaechentreagheitsmomente}. Das Flächenträgheitsmoment eines Stabs mit quadratischen Querschnitt und Durchmesser $d$ ist durch
\begin{align}
    \label{eqn:kreis}
    I_{\symup{\bigcirc}} = \frac{\symup{\pi}d^4}{64}
\end{align}
gegeben \cite{flaechentreagheitsmomente}.
