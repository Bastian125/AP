\section{Theorie}
\label{sec:Theorie}
\subsection{Das Trägheitsmoment und Drehmoment}
\subsubsection{Das Trägheitsmoment}
Bei einer geradlinigen Bewegung wird eine Änderung der Bahn durch die Kraft $\vec{F} = m \cdot \vec{a}$ bewirkt.
Diese Kraft hängt von der Masse $m$ ab. Diese Masse ist die Trägheit eines starren Körpers gegenüber der Änderung
der Geschwindigkeit. 
\\
Analog dazu gibt es bei der Rotation das Drehmoment und das Trägheitsmoment. Das Trägheitsmoment gibt also (wie die 
Masse bei der geradlinigen Bewegung) an, wie träge ein Körper gegenüber einer Änderung der Winkelgeschwindigkeit ist.
Im Folgenden werden wir das Formelzeichen $I$ für das Trägheitsmoment verwenden.
\\
Allgemein wird das Gesamtträgheitsmoment eines ausgedehnten Körpers folgendermaßen bestimmt:
\begin{align}
    I = \sum_{i}^{n} r_i^2 \cdot m_i
\end{align}
$r_i$ ist hierbei der Abstand der Massenelemente $m_i$ von der Drehachse.
Für infinitisimale Massen integrieren wir:
\begin{align}
    I = \int_{}^{}r^2 \mathrm{d}m
\end{align}
Aus den Formeln folgt bereits, dass das Trägheitsmoment $I$ im Gegensatz zu der Masse $m$ nicht in $\mathrm{kg}$ angegeben wird,
sondern in $\mathrm{kg} \cdot \mathrm{m}^2$.
\\ \\
Das Trägheitsmoment wird immer bezüglich einer Drehachse angegeben. Falls diese Achse nicht durch den Schwerpunkt des Körpers
verläuft, sondern parallel mit einem Abstand $a$, so kann man mithilfe des Steinerschen Satzes das Trägheitsmoment bezüglich
der verschobenen Achse berechnen:
\begin{equation}
    I = I_S + m \cdot a^2
\end{equation}
$I_S$ ist das Trägheitsmoment durch den Schwerpunkt des Körpers, $m$ die Masse des Körpers und $a$, wie oben erwähnt, ist der 
Abstand der beiden Achsen.
\subsubsection{Das Drehmoment}
Das Drehmoment ist das Analogon zu der Kraft bei einer geradlinigen Bewegung und wird so bestimmt:
\begin{align}
    \vec{M} = \vec{F} \times \vec{r} \:\: \mathrm{bzw.} \:\: M = F \cdot r \cdot sin(\phi)
\end{align}
Wenn der Körper durch das Drehmoment aus seiner Ruhelage ausgelenkt wird, führt das bei schwingungsfähigen Systemen dazu,
dass ein rücktreibendes Drehmoment (z.B. durch eine Feder) bewirkt wird und der Körper anfängt harmonisch zu schwingen. Die Schwingungdsdauer
ist gegeben durch:
\begin{align}
    T = 2\pi \sqrt{\frac{I}{D}}
\end{align}
Dabei ist $D$ die Winkelrichtgröße und $I$ das Gesamtträgheitsmoment. Mit der Winkelrichtgröße lässt sich auch das Drehmoment beschreiben: $M = D \cdot \phi$.

\cite{sample}
\newpage