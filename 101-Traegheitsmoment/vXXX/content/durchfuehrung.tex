\section{Durchführung}
\label{sec:Durchführung}
Auf einer zweifach mit einem Rahmen verbundenen Drillachse werden unterschiedliche Körper befestigt.
Die Drillachse ist durch eine Feder mit dem Rahmen verbunden. Um später Trägheitsmomente zu bestimmen,
muss die Federkonstante und das Eigenträgheitsmoment der Drillachse bestimmt werden. \\
Die Federkonstante D wird durch Ansetzen einer Federwage an einem Stab, der als masselos angenommen werden kann, 
in einem Abstand r zur Drillachse bestimmt. Für zehn Auslenkungen der Stange mit unterschiedlichen Winkeln $\phi$ wird eine Kraft $F$ gemessen.\\
Das Eigenträgheitsmoment $I_D$ wird durch Anbringen von zwei Zylindern im gleichen Abstand von der Drillachse
an der Stange gemessen. Dabei wird die Stange durch Auslenkung in Schwingung gebracht und mit einer Stoppuhr
die Schwingungsdauer gemessen. Hier wird die Stange 10 Mal ausgelenkt unter demselben Auslenkwinkel $\phi$.\\
Im Anschluss wird das Trägheitsmoment eines Zylinders und einer Kugel bestimmt. Dies geschieht wieder durch 
Auslenkung der Drillachse, so dass eine Schwingungsdauer gemessen werden kann. Erneut wird die Stange 10 Mal und diesmal unter
dem Winkel $\phi= 90^{\circ}$ ausgelenkt.\\
Nach dem gleichen Prinzip wird das Trägheitsmoment einer Holzfigur in zwei Positionen bestimmt.
In der ersten Position sind die Beine der Figur ausgestreckt und in der zweiten Position sind die Arme ausgestreckt.
In beiden Positionen wird die Figur 10 Mal ausgelenkt unter einem Winkel von $\phi=90^{\circ}$.\\
Bei allen Versuchen wird ein Vielfaches der Periodendauer gemessen und anschließend durch die Anzahl der
Perioden geteilt.
\newpage