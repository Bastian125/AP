\section{Auswertung}
\label{sec:Auswertung}
\subsection{Winkelrichtgröße}
Die Winkelrichtgröße wird durch die Formel
\begin{equation}
  D = \frac{F \cdot r}{\phi}
\end{equation}
bestimmt. Die verwendeten Werte sind in \ref{tab:winkelrichtgr} angegeben.
\begin{table}
  \centering
  \caption{Messdaten zur Bestimmung der Winkelrichtgröße D}
  \label{tab:winkelrichtgr}
  \begin{tabular}{c c c c}
    \toprule
    $F/N$ & $\phi$ & $r/m$ & $D/Nm$ \\
    \midrule
    0,1  &  30 & 0,1 & 0,000333\\
    0,26 &  60 & 0,1 & 0,000433\\
    0,41 &  90 & 0,1 & 0,000456\\
    0,56 & 120 & 0,1 & 0,000467\\
    0,72 & 150 & 0,1 & 0,000480\\
    0,85 & 180 & 0,1 & 0,000472\\
    0,48 & 180 & 0,2 & 0,000533\\
    0,55 & 240 & 0,2 & 0,000458\\
    0,63 & 270 & 0,2 & 0,000467\\
    0,69 & 300 & 0,2 & 0,000460\\
    \bottomrule
  \end{tabular}
\end{table}
\\Sowohl der Mittelwert, als auch die Standardabweichung wurden mit Python bestimmt. Daraus ergibt sich der
gemittelte Wert
\begin{align*}
    D = (0{,}000456 \pm 0{,}000048)\,\mathrm{Nm} .
\end{align*}

\subsection{Eigenträgheitsmoment}
% \begin{figure}
%   \centering
%   \includegraphics{plot.pdf}
%   \caption{Plot.}
%   \label{fig:plot}
% \end{figure}

\subsection{Trägheitsmoment des Zylinders}
\subsubsection{Theoretische Werte}

\subsubsection{Experimentelle Werte}
Der Zylinder wird auf der Drillachse um den Winkel $\phi_{Zyl} = 90^{\circ}$ ausgelenkt und die Zeit
nach fünf Schwingungen gestoppt.
Durch teilen der Zeitmessungen $Z_{Zyl}$ durch fünf ergeben sich die Schwingungsdauern $T_{Zyl}$. 
Diese sind in Tabelle \ref{tab:T_Zyl} zu finden.
\begin{table}
  \centering
  \caption{Messdaten der Schwingungsdauer des Zylinders}
  \label{tab:T_Zyl}
  \begin{tabular}{c c}
    \toprule
    $Z_{Zyl}$ & $T_{Zyl}/s$ \\
    \midrule
    3.94 & 0.79 \\
    3.75 & 0.75 \\
    4.16 & 0.83 \\
    5.78 & 1.16 \\
    3.69 & 0.74 \\
    3.97 & 0.79 \\
    3.85 & 0.77 \\
    3.84 & 0.77 \\
    4.12 & 0.82 \\
    3.88 & 0.78 \\
    \bottomrule
  \end{tabular}
\end{table}
\\
Der Mittelwert und die Abweichung wurden wieder mit Python berechnet.
Aus den Daten ergibt sich
\begin{align*}
  T_{Zyl} = (0{,}82 \pm 0{,}12)\, \mathrm{s} .
\end{align*}

\subsection{Trägheitsmoment der Kugel}
\subsubsection{Theoretische Werte}

\subsubsection{Experimentelle Werte}


%Siehe \autoref{fig:plot}!
