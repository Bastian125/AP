\section{Diskussion}
\label{sec:Diskussion}

Durch die ersten drei Versuchsteile konnten drei verschiedene Werte für Die Zeitkonstante
$RC$ ermittelt werden: \\
\begin{align*}
    RC_1 &= (13{,}245 \pm 0{,}063) \, \cdot \, \si{\micro s} &
	     &\text{Bestimmung durch Entladekurve}
	\\
	RC_2 &= (61,2 \pm 9)  \, \cdot \, \si{\micro s} &
	     &\text{Bestimmung durch das Amplitudenverhältnis}
	\\
	RC_3 &= (53,7 \pm 0,0002) \, \cdot \, \si{\micro s} &
	     &\text{Bestimmung durch die Phasenverschiebung}
\end{align*}

Auffällig ist hier, dass der erste Wert deutlich von den anderen beiden Werten abweicht. Das kann durch Messfehler
oder durch Fehler der Messappartatur zu begründen sein. Da die anderen beiden Werte nur um $12,3 \%$ voneinander abweichen,
ist es wahrscheinlich, dass der erste Wert falsch ist.\\
Erfreulich ist jedoch, dass die experimentell ermittelten Werte immer sehr gut zu den theoretisch ermittelten 
passen und die Ausgleichsgeraden nur kleine Abweichuungen zu den Messwerten haben.\\
Allerdings muss bei der Versuchsdurchführung des zweiten Versuchteils ein Fehler passiert sein. Die Amplitude
der Speisespannung wurde nämlich besonders bei kleinen Frequenzen kleiner gemessen, als die Amplitude der 
Kondensatorspannung. Hier liegt entweder ein Fehler des Oszilloskops oder ein Fehler beim Ablesen der Werte vor.
Vewunderlicherweise passen die gemessenen Werte trotzdem gut zu den Theoriewerten.