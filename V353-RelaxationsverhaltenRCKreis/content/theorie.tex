\section{Zielsetzung}
\label{sec:Zielsetzung}
Es soll das Relaxationsverhalten des Entladevorgangs eines RC-Kreises bestimmt werden.

\section{Theorie}
\label{sec:Theorie}

\subsection{Allgemeine Relaxationsgleichung}
\label{sec:AllgemeineRelaxationsgleichung}
Es handelt sich um Relaxationverhalten, wenn ein System aus einem Ausgangszustand ausgelenkt wird und ohne Oszillation in denselben
Zustand zurückkehrt. Allgemein lässt sich eine Differentialgleichung der Form
\begin{align}
    \label{eqn:dA/dt}
    \frac{\symup{d}A}{\symup{d}t} = c[A(t)-A(\infty)]
\end{align}
für die Änderungsgeschwindigkeit der Größe $A$ aufstellen. Diese lässt sich durch Umformung lösen zu
\begin{align}
    \label{eqn:AllgemeineRelaxationsgleichung}
    A(t)=A(\infty)+[A(t)-A(\infty)]e^{ct}.
\end{align}

\subsection{Entladevorgang eines Kondensators}
\label{sec:EntladekurveeinesKondensators}
\begin{align}
    \label{eqn:Entladung}
    Q(t)=Q(0)e^{-\frac{1}{RC}}
\end{align}

\subsection{Relaxationsverhalten bei angelegter Wechselspannung}
\label{sec:RelaxationsverhaltenbeiangelegterWechselspannung}
\begin{align}
    \label{eqn:Amplitude}
    A(\omega)=\frac{U_0}{\sqrt{1+\omega^2 R^2 C^2}}
\end{align}

\subsection{Integrationsverhalten eines RC-Kreises}
\label{sec:IntegrationsverhalteneinesRC-Kreises}
\begin{align}
    \label{eqn:Kondensatorspannung}
    U_{C}(t)=\frac{1}{RC}\int_{0}^{t} U(t')\symup{d}t'
\end{align}

\cite{sample}
