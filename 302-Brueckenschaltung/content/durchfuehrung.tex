\section{Durchführung}
\label{sec:Durchführung}
Alle Brückenschaltungen werden mit einer Wechselspannung mit 1000 Hz betrieben und als Spannungsmessgerät wird
ein digitales Oszilloskop benutzt.

\subsection{Wheatstonesche Brückenschaltung}

Die Schaltung wird nach \autoref{Abb:Wheatstone} aufgebaut und sich zwei unbekannte Widerstände ausgesucht. In diesem
Fall wurden Wert 13 und Wert 14 verwendet. Für Wert 13 wird das Potentiometer zunächst solange verstellt, bis die gemessene Brückenspannung
verschwindet. Dann werden alle bekannten Werte abgelesen und notiert. Dies wird für jeweils drei verschiedene bekannte 
Widerstände $R_2$ wiederholt. Anschließend wird das Ganze nochmal für Wert 14 durchgeführt.

\subsection{Kapazitätsmessbrücke}

Die Schaltung wird zunächst wie in \autoref{Abb:Kapazitaetsmessbruecke} aufgebaut und sich zwei Kondensatoren mit unbekannten
Kapazitäten und in Reihe geschalteten unbekannten Widerständen rausgesucht. Hier wurde nur Wert 8 verwendet, da Wert 15 fehlerhafte Ergebnisse
am Oszilloskop angezeigt hatte. \\
Das Potentiometer und der veränderliche Widerstand $R_2$ werden so lange verändert, bis die Brückenspannung minimal ist. Alle bekannten 
Werte werden wieder abgelesen und notiert.

\subsection{Induktivitätsmessbrücke}

Die Schaltung wird nach \autoref{Abb:Induktivitaetsmessbruecke} aufgebaut und analog zur Kapazitätsmessbrücke durchgeführt. Nur, dass
anstelle des Kondensators $C_2$ nun eine bekannte Spule $L_2$ verwendet wird. Die Werte werden wieder abgelesen und notiert.

\subsection{Maxwell-Brücke}

Die Brückenschaltung wird wie in \autoref{Abb:Maxwell} aufgebaut und sich dasselbe unbekannte Bauteil wie bei der Induktivitätsmessbrücke genommen,
welches aus einer unbekannten Spule und einem unbekannten Ohm'schen Widerstand besteht. Die Widerstände $R_3$ und $R_4$ werden so lange varriiert, 
bis die gemessene Brückenspannung Null wird. Dann werden wieder alle Werte notiert.

\subsection{Wien-Robinson-Brücke}

Die Schaltung wird wie in \autoref{Abb:Wien} gezeigt aufgebaut. Bei diesem Versuch sind alle verwendeten Bauteile bekannt. Es wird die Spannungsfrequenz 
zwischen 20 und 30 000 Hz varriiert und notiert wie sich die Brückenspannung $U_{Br}$ dementsprechend verändert. Hierzu wurde bei 20 Hz begonnen und der
nächste Wert immer als doppeltes des vorherigen genommen, also: 20 Hz, 40 Hz, 80 Hz... . Anschließend wurde die Speisespannung $U_S$ nach dem gleichen Schema untersucht.

\newpage