\section{Auswertung}
\label{sec:Auswertung}

\subsection{Wheatston'sche Messbrücke}

\begin{table}
  \centering
  \caption{Messung von $R_3$ und $R_4$ für $R_{14}$}
  \label{tab:R14}
  \begin{tabular}{c c c c}
    \toprule
    $R_2/\unit{\ohm}$ & $R_3/\unit{\ohm}$ & $R_4/\unit{\ohm}$ & $R_{14}/\unit{\ohm}$ \\
    \midrule
     332 & 243 & 757 &  106,6 \\
     664 & 392 & 608 &  428,1 \\
    1000 & 612 & 388 & 1577,3 \\
    \bottomrule
  \end{tabular}
\end{table}

\begin{table}
  \centering
  \caption{Messung von $R_3$ und $R_4$ für $R_{13}$}
  \label{tab:R13}
  \begin{tabular}{c c c c}
    \toprule
    $R_2/\unit{\ohm}$ & $R_3/\unit{\ohm}$ & $R_4/\unit{\ohm}$ & $R_{13}/\unit{\ohm}$ \\
    \midrule
     332 & 579 & 421 &  456,6 \\
     664 & 595 & 405 &  975,5 \\
    1000 & 789 & 211 & 3739,3 \\
    \bottomrule
  \end{tabular}
\end{table}

\subsection{Kapazitätsmessbrücke}
\begin{table}
  \centering
  \caption{Messung von $C_x$ und $R_x$}
  \label{tab:Cx,Rx}
  \begin{tabular}{c c c c c}
    \toprule
    $R_2/\unit{\ohm}$ & $R_3/\unit{\ohm}$ & $R_4/\unit{\ohm}$ & $C_x/10^(-9)\unit{\farad}$ & $R_x/\unit{\ohm}$ \\
    \midrule
     500 & 640 & 360 & 336 & 889 \\
     600 & 580 & 420 & 432 & 829 \\
     700 & 480 & 520 & 647 & 646 \\
     800 & 491 & 509 & 619 & 772 \\
     900 & 470 & 530 & 673 & 789 \\
    1000 & 440 & 560 & 760 & 786 \\
    \bottomrule
  \end{tabular}
\end{table}


\begin{figure}
  \centering
  \includegraphics{plot.pdf}
  \caption{Plot.}
  \label{fig:plot}
\end{figure}


Siehe \autoref{fig:plot}!
